\documentclass[review, 3p, times]{elsarticle} %final, twocolumn, 5p, 12pt, authoryear, twocolumn, 5p,

\usepackage[utf8]{inputenc}
\usepackage[T1]{fontenc}

\usepackage{amsmath}
\usepackage{amsfonts}
\usepackage{amssymb}

% improved float environment
\usepackage{float}
% improved underlines (unterstreichungen)
\usepackage[normalem]{ulem}

% plotting
\usepackage{tikz}
\usepackage{pgfplots}
\pgfplotsset{compat=1.17}

% tables
\usepackage{booktabs}
\usepackage{tabulary}
\usepackage{threeparttable}
\usepackage{longtable}
%\usepackage{ctable}
\usepackage{makecell}

% SI units
\usepackage{siunitx}
\DeclareSIUnit{\thermal}{th}
\DeclareSIUnit{\electric}{el}
\DeclareSIUnit{\year}{a}

% Euro symbol
\usepackage[gen]{eurosym}

% citations
\usepackage{natbib}

% to-do notes
\setlength {\marginparwidth }{2cm}
\usepackage[draft]{todonotes}

% graphics and figures
\usepackage{graphicx}
\usepackage{caption}
\usepackage{subcaption}

\DeclareMathOperator*{\argmax}{arg\,max}
\newcommand{\COO}{\ensuremath{\mathrm{CO_2}} }

\usepackage[colorlinks=true, citecolor=blue, linkcolor=blue, filecolor=blue,urlcolor=blue]{hyperref}

\begin{document}
    \begin{frontmatter}
        \title{The Cost of Undisturbed Landscapes}
        \author[1]{Sebastian Wehrle\corref{cor1}}
        \ead{sebastian.wehrle@boku.ac.at}
        \author[1]{Johannes Schmidt}
        \ead{johannes.schmidt@boku.ac.at}
        \cortext[cor1]{Corresponding author}
        \address[1]{Institute for Sustainable Economic Development, University of Natural Resources and Life Sciences,
        Feistmantelstrasse 4, 1180 Vienna, Austria}

        \begin{abstract}
            Lorem ipsum et dolores
        \end{abstract}

        \begin{keyword}
            externalities \sep wind power \sep system design \sep policy
            \JEL L94 \sep Q41 \sep C61
        \end{keyword}
    \end{frontmatter}
    \newpage

% %%%%% %%%%% %%%%% DATA AND METHODS %%%%% %%%%% %%%%% %%%%%


    \section{Introduction} \label{sec:introduction}
    Future, low carbon energy systems are likely to largely rely on electricity from renewable energy technologies
    (RET), such as solar photovoltaics (PV) or wind turbines, according to a substantial body of
    literature~\citep[e.g.][]{Becker2014,Delucchi2011,Fernandes2014,Gils2017,Jacobson2011,Schmidt2016}.
    \todo{references not good for statement}
    A broad stream of papers~\citep[e.g.][]{Brown2018,Rodriguez2015,Schlachtberger2018,Scholz2017} investigates
    conditions under which total system cost, conceived as the sum of fuel and emission cost, operation and maintenance
    cost, and investment cost, can be minimized.
    In this context, available resource potentials are an important boundary condition for the determination of such
    least-cost system configurations.
    Fortunately, such resource potentials are assessed by numerous scholars at various levels.
    Frequently, these assessments combine indicators of the wind resource with technical or legal restrictions on land
    available for RET deployment~\citep[e.g.][]{Bosch2017,McKenna2015,Ryberg2020}.
    However, there is an increasing awareness that such approaches are not a proper reflection of deployable RET
    capacites, as a more nuanced analysis would have to account for the social acceptance of RET.
    Opposition to wind turbines, for example, is often based on perceived negative interference with
    landscapes~\cite{Mattmann2016}, potential harm to wildlife~\cite{Loss2013, Voigt2015, Wang2015}, or
    noise~\cite{Wang2015a}.
    Zerrahn~\cite{Zerrahn2017} provides an excellent survey of the literature on wind power externalities.

    Based on this insight, several contributions sought to take social acceptance into account when assessing RET
    potentials.
    Höltinger et al.\ \cite{Hoeltinger2016}, for example, relied on participatory modeling to identify uncontested wind
    power potentials in Austria.
    Jäger et al.\ \cite{Jaeger2016} \todo[inline]{brief description of paper}
    Harper et al.\ \cite{Harper2019} develop a spatially resolved multi-criteria analysis including technological, legal,
    and social constraints to determine sites suitable for wind turbine erection.
    % mention minimum distances to settlements

    In contrast to the determination of such `augmented potentials' trying to identify sites that are certainly suited
    for RET deployment, Drechsler et al.\ \cite{Drechsler2011} argue that a welfare-optimal spatial allocation of RET
    needs to consider social costs and benefits.
    Social conflict would then be mitigated by selecting sites with the lowest welfare loss.
    To demonstrate their approach, Drechsler et al.\ conduct a case study for a region in Germany, where they survey
    preferences of inhabitants to explicitly consider the trade-off between \emph{social cost} arising from negative
    externalities of wind power and electricity generation cost of wind power due to spatially varying wind conditions.
    Some of the authors studied implications of welfare optimal allocation of wind turbines also in more recent
    contributions.
    Salomon et al. (2020)~\citep{Salomon2020} study welfare-optimal rules on minimum distances between wind turbines and
    human settlements.
    Drechsler et al. (2017)~\citep{Drechsler2017} study welfare-optimal allocation of wind turbines and solar PV.\todo[inline]{expand on this}

    We expand on this work by conceptualising welfare optimal RET deployment in more general settings, so that we can
    account for the opportunity cost of RET substitution (arising from different system integration cost of RET), as
    well as spatial differences in endowments with renewable resources and negative externalities from RET depending on
    spatially diverse characteristics, as we lay out in section~\ref{subsec:framework} below.
    \todo[inline]{how? accounting for everything could be done by spatially resolved model that has info on resource
    availability and on negative impact for each location.
    Alternatively, use model for opportunity cost only and calc deviation against mean resource?}

    To undergird our proposed methodology, we use a numerical power system model (see
    section~\ref{sec:data-and-methods}) to investigate the trade-off between deploying different renewable energy
    technologies in terms of system cost, which constitute an important, though incomplete part of social cost.
    We do this to underline the importance of RET substitution and the corresponding opportunity costs in the
    identification of spatial allocations of RET as well as in power system planning in general.
    In the latter context, we also contribute to the analysis of near-optimal power system
    configurations~\citep[see e.g.][]{Neumann2019,Schlachtberger2017}.

    % What do we want to show?
    % We embed our demonstration in a wider framework suited to expand renewable energy technologies at minimal social cost,
    % thereby mitigating potential social conflict over the transition towards low-carbon energy systems.
    % The current contribution demonstrates that:
    % - there is a tradeoff between different RET

    \subsection{Framework} \label{subsec:framework}
    Similar to~\cite{Drechsler2011}, we regard the problem as one of welfare optimization, in which we seek to minimize
    the social cost (including, but not limited to system cost) of low carbon energy systems.
    Our framework is best understood by considering a benevolent social planner who is to decide on the deployment of
    RET\@.
    The social planner faces a trade-off between
    (i) (potential) negative externalities of a RET at the local level (recall that externalities from emissions are
    already included in system cost), such as the visual impact on landscapes and
    (ii) (potential) positive impact of a RET through lowering system cost (compared to its best alternative).

    \begin{figure}[H]
        \centering
        \begin{tikzpicture}
            \begin{axis}[
                width=0.66 \textwidth,
                clip=false,
                axis lines=left,
                ticks=none,
                xlabel={$\alpha$},
                ylabel={cost},
                domain=0.5:1.75,
                xmin=0.5, xmax=2,
                ymin=0, ymax=4,
                samples=200
            ]
                \addplot[thick] (x, x^2-0.25);
                \addplot[thick] (x, x^2-4.25*x+5);
                \node at (axis cs: 1.8,2.85) {$s$};
                \node at (axis cs: 1.8,0.65) {$h$};
                \node at (axis cs: 0.45,-0.1) {$0$};
                %\addplot (1.235, x);
            \end{axis}
        \end{tikzpicture}
        \caption{Social optimum}
        \label{figure:social-optimum}
    \end{figure}

    For illustration, consider figure~\ref{figure:social-optimum}, which depicts the effect of deviating from a
    system-cost minimizing RET allocation by substituting some share $\alpha$ of technology $A$ with its best alternative,
    technology $B$.\footnote{Here, we consider the interesting case in which in system-cost minimum a RET causes higher
    external cost than it saves in system cost.}
    We let $\alpha$ denote the share of technology $A$ substituted by the best available alternative technology $B$.
    At $\alpha=0$, we are in the system cost minimum, while the 'harm' $h$ from otherwise unaccounted external cost is
    high.
    Now suppose that we can lower external cost by substituting $A$ with $B$, i.e.\ harm $h$ is decreasing in $\alpha$.
    A benevolent social planner will substitute $A$ with $B$ as long as the resulting increase in system cost is lower
    than the corresponding decline in harm.

    \subsection{Case study}
    Based on this framework, we use the case of Austria to quantify the opportunity cost of wind power (line $s$ in
    figure~\ref{figure:social-optimum}).
    We do not set out to quantify harm (i.e.\ negative external effects) induced by wind power.
    While such an endeavor is beyond the scope of this paper, it is a frequent undertaking in the
    literature~\citep[e.g.][]{Droees2016, Gibbons2015, Heintzelmann2017, Jensen2018, Kussel2019, Lang2014, Sims2008, Sunak2016, Vyn2014},
    even though an assessment of negative wind turbine externalities has to the best of our knowledge not been conducted
    for Austria yet.

    We chose Austria as our case to study, as the government has committed itself to meeting \SI{100}{\percent} of
    electricity consumption from renewable sources (on annual balance, excluding industry own-consumption and system
    services) by 2030.
    This policy makes Austria an ideal case for studying the substitution of RET while keeping total renewable
    electricity generation constant.
    Moreover, in Austria the bulk of additional renewable electricity must come either from wind power or solar
    photovoltaics, as other renewables potentials are largely exploited already (see
    section~\ref{subsubsec:assumptions-austria:}), which simplifies our analysis further.

    In the context of Austria, we find a least-cost renewable electricity system to be largely based on wind power,
    which allows us to quantify the opportunity cost of wind power in terms of its best alternative solar PV.
    Corresponding results are given in section~\ref{sec:results-discussion}.

% %%%%% %%%%% %%%%% DATA AND METHODS %%%%% %%%%% %%%%% %%%%%


    \section{Data and Methods} \label{sec:data-and-methods}
    To investigate the cost of undisturbed landscapes (i.e. the opportunity cost of wind power) we use the power system
    model \emph{medea}, summarized in section~\ref{subsec:medea}, which we instantiate with scenario assumptions
    and observed data of the year 2016 (see section~\ref{subsec:data}).
    For determining the opportunity cost of wind power, we proceed as follows:
    \begin{enumerate}
        \item We derive the unrestricted system cost-minimizing deployment of wind and solar power,
        \item we restrict deployment of wind power by a small margin (so that the next best RET substitutes for wind power)
        and observe net system cost $c_{net}$ for Austria (calculated as total system cost including air pollution cost
        net of trade balance).\label{enumerate:restrict}
        \item We repeat step~\ref{enumerate:restrict} till no wind power can be deployed.
        \item Finally, we approximate the opportunity cost of wind power (at given wind power capacity $w$) $OC_w$ by
        the change in $c_{net}$ in response to a change in wind power capacity $w$ deployed, i.e.\
        \[OC_w = \frac{\Delta c_{net}}{\Delta w}\]
        where $\Delta$ is the difference operator.
    \end{enumerate}

% ***** medea *****
    \subsection{Power system model \emph{medea}} \label{subsec:medea}
    We make use of the power system model \emph{medea} to simulate (dis)investment in and hourly operation of the
    prospective Austrian (and German) power system.
    The model is cast as a linear optimization seeking to minimize total system cost, which consist of fuel and emission
    cost, quasi-fixed and variable operation and maintenance (O\&M) cost, the costs of investment in energy generation,
    storage, and transmission assets, and (potential) cost of non-served load.
    In terms of economics, the model reflects a perfectly competitive energy-only market with fully price-inelastic
    demand and perfect foresight of all actors.

    The system is required to meet exogenous and inelastic demand for electricity and heat in any hour of the modeled
    year.
    Energy supply, in turn, is constrained by available installed capacities of energy conversion, storage, and
    transmission units.
    Co-generation units convert fuel to heat and power subject to a feasible operating region defined by the unit's
    electrical efficiency, the electricity loss per unit of heat production, and the back pressure coefficient.
    Electricity generation from intermittent sources (wind, run-of-river hydro, solar) is subject to exogenous hourly
    generation profiles, which are scaled according to total installed capacities.
    Electricity from these sources can be curtailed at no additional cost (free disposal).

    Electricity can be stored in reservoir and pumped hydro storages or batteries.
    The capacity of hydro storages can not be expanded, as we assume existing potentials to be exhausted.
    Battery capacity, on the other hand, can be added endogenously.
    Generation from storages is constrained by installed capacity and energy contained in storage.
    Hydro reservoirs are filled by exogenously given inflows.
    Pumped hydro storages and batteries can actively store electricity for later use.
    To better capture operational differences of hydro storage units, we model daily, weekly and seasonal reservoir and
    pumped storage plants separately.
    To ensure a stable and secure operation of the electricity system, ancillary services (e.g. frequency control,
    voltage support) are required.
    We model ancillary service needs as a minimum requirement on spinning reserves operating at any point in time.
    Thus, we assume that ancillary services can be provided by thermal power plant, run-of-river hydro plant or any
    type of storage technology.
    We do not model transmission and distribution grids within Austria or Germany.
    We do, however, account for the congestion management scheme introduced by German and Austrian authorities in 2018,
    that limits cross-border electricity trade.
    \todo[inline]{is this elegant? Should i mention that investment in x-border capacity is allowed but does not occur?}
    For a detailed, mathematical description please refer to \ref{sec:medea}.

    Data processing is implemented in python, while the optimization model is based on GAMS.
    Running the model on an Intel i7-8700 machine with 16 GB RAM, using CPLEX 12.9 as a solver, takes 10--15 minutes,
    depending on the model parameters.

% ***** Data *****

    \subsection{Data}\label{subsec:data}
    We set up our model with the goal to resemble Austria's prospective electricity and district heating systems in the
    year 2030.
    In addition, we model Austria's largest electricity trading partner Germany, to account for potential effects from
    electricity trade.
    As for Austria, our scenario reflects Germany's currently announced electricity sector policies, so that we set
    generation capacities to levels consistent with policies effective by 2030.
    These assumptions are laid out in sections~\ref{subsubsec:assumptions-austria} and \ref{subsubsec:assumptions-germany}
    and summarized in table~\ref{tab:instcaps}.

    \subsubsection{Scenario Assumptions for Austria}\label{subsubsec:assumptions-austria}
    The Austrian government has set itself the goal of generating \SI{100}{\percent} of electricity consumption from
    renewable sources on annual balance by 2030~\citep{Regierungsprogramm2020}.
    However, industry own consumption and system services, which currently account for about \SI{10}{\percent} of annual
    electricity consumption, are exempt.
    The government plans to achieve this goal by generating an additional \SI{27}{\tera\watt\hour} of electricity
    annually from renewable sources\footnote{Please note that this policy goal does not imply that the generated
    electricity must actually be consumed. Hence, we count curtailed electricity as contributing to the policy goal.}.
    New hydro power plants are thought to contribute \SI{5}{\tera\watt\hour} annually, while additional electricity
    generation from biomass is envisaged to account for \SI{1}{\tera\watt\hour} annually.
    We add corresponding generation capacities to reach these targets.
    The government projects the remainder to come from solar PV (\SI{11}{\tera\watt\hour} annually) and onshore wind
    turbines (\SI{10}{\tera\watt\hour} annually)~\cite{Regierungsprogramm2020}.
    As we are interested in determining the opportunity cost of wind power versus its best alternative, we allow for
    endogenous investment in wind and solar power, without enforcing announced targets.
    Other low-carbon energy technologies are not feasible at large scale in the Austrian context.
    The use of nuclear power was banned in a 1978 referendum~\citep{Pelinka1983}.
    Biomass lacks ecological as well as economical sustainability~\citep{Erb2018}, and hydro power potentials were
    already largely exploited~\citep{Walder2014} even before government plans of adding another \SI{5}{\tera\watt\hour}
    in annual hydro power generation.
    Initial generation capacities for Austria are given in table~\ref{tab:instcaps}.

    \subsubsection{Scenario Assumptions for Germany}\label{subsubsec:assumptions-germany}
    Germany has announced specific capacity targets for several power generation technologies.
    Following these announcements, we anticipate an end to nuclear power generation, a (partial) coal exit according to
    recommendations by the `coal commission'~\cite{WSB2019}, and a further expansion of renewable electricity generation
    in line with the German Renewable Energy Sources Act~\cite{noauthor_renewable_2017}.
    We do, however, expect the \SI{52}{\giga\watt} solar PV limit to be lifted.
    The corresponding capacity assumptions are displayed in table~\ref{tab:instcaps}.

    \begin{table}[H]
        \caption{Initial Generation Capacities}
        \centering
        \begin{tabulary}{0.55\textwidth}{lcc}
            \toprule
            Technology & Austria & Germany \\
            & GW & GW \\
            \midrule
            Wind Onshore & $2.6$ & $90.8$ \\
            Wind Offshore & $0.0$ & $15.0$ \\
            Solar PV & $1.1$ & $73.0$ \\
            Run-of-river Hydro & $6.9$ & $4.5$ \\
            Biomass & $0.725$ & $8.4$ \\
            Lignite & $0.0$ & $11.4$ \\
            Coal & $0.0$ & $14.0$ \\
            Natural Gas & $3.9$ & $24.2$ \\
            Mineral Oil & $0.2$ & $3.5$ \\
            Heat Pump & $1.0$ & $1.0$ \\
            Gas Boiler & $3.9$ & $25.5$ \\ \bottomrule
        \end{tabulary}
        \label{tab:instcaps}
    \end{table}

    \subsection{Energy supply}
    We represent 27\todo{compare with table!} dispatchable energy conversion and storage technologies that are expected
    to be operated in 2030.
    In addition to given, initally installed capacities (see~\ref{tab:instcaps}), the model can endogenously add further
    generation capacities that are compatible with stated policy objectives.
    All technologies can also be decommissioned, so that we adopt a long-run perspective on the power system.
    Parameters of admissible technologies are summarized in Table~\ref{itm:techcost}.

    \begin{table*}[h!t]
        \centering
        \begin{threeparttable}
            \caption{Technology Input Parameters}\label{itm:techcost}
            %\begin{longtable}{p{2.9cm} p{1.5cm} p{1.5cm} p{1.6cm} p{1.5cm} p{1.0cm} p{1.2cm} p{1.0cm}}
            \begin{tabulary}{0.5\textwidth}{lccccccl}
                \toprule
                & \multicolumn{2}{c}{Capital Cost} & \multicolumn{2}{c}{O\&M-Cost} & & & \\ \cmidrule(lr){2-3} \cmidrule(lr){4-5}
                Technology & Power & Energy & Quasi- & Variable & Life- & Efficiency & Source    \\
                & & & fixed & & time & &           \\
                & \EUR/kW & \EUR/kWh & \EUR/MWa & \EUR/MWh & a & & \\
                \midrule
                %\multicolumn{8}{c}{\emph{Renewable Energy Technologies}}\\ \midrule
                Wind Onshore & $1040$    & NA & $12600$   & $1.35$     & $30$ & NaN & 1 \\
                Wind Offshore & $1930$    & NA & $36053$   & $2.70$     & $30$ & NaN & 1 \\
                %Solar PV       & $630$ & NaN & $8844$ & $0.00$ & $40$ & NaN & 9 \\
                PV Rooftop & $870$     & NA & $10815$   & $0.00$     & $40$ & NaN & 1 \\
                PV Open-space & $330$     & NA & $6380$    & $0.00$     & $40$ & NaN & 1 \\ \midrule
                %\multicolumn{8}{c}{\emph{Thermal Energy Conversion}}\\ \midrule
                Lignite Adv & -- & NA & $40500$   & $0.85$     & $40$ & 0.439 & \cite{OekoInstitut2017} \\
                Coal SC & -- & NA & $25000$   & $6.00$     & $40$ & 0.425 & 1 \\
                Coal USC & -- & NA & $31000$   & $2.90$     & $40$ & 0.485 & 1 \\
                Nat Gas ST & $400$     & NA & $15000$   & $3.00$     & $30$ & 0.407 & 1 \\
                Nat Gas GT & $590$     & NA & $19500$   & $4.40$     & $25$ & 0.420 & 1 \\
                Nat Gas CC & $880$     & NA & $29300$   & $4.40$     & $25$ & 0.610 & 1 \\
                Oil ST & $400$     & NA & $6000$    & $3.00$     & $30$ & 0.396 & 1 \\
                Oil GT & $378$     & NA & $8068$    & $4.50$     & $25$ & 0.410 & 1 \\
                Oil CC & $800$     & NA & $25000$   & $4.00$     & $25$ & 0.470 & 1 \\
                Biomass & $3400$    & NA & $140000$  & $4.50$     & $25$ & 0.298 & 1 \\ \midrule
                %\multicolumn{8}{c}{\emph{Energy Storage Technologies}}\\ \midrule
                Hydro Reservoir & -- & -- & $X$       & $X$       & $80$ & $0.9$  & X \\
                Hydro PSP & -- & -- & $X$       & $X$       & $80$ & $0.9^2$ & X\\
                Battery Li-Ion & $1930$    & $321$ & $36053$   & $2.70$    & $30$ & NaN & 1 \\ \midrule
                %\multicolumn{8}{c}{\emph{Heat Generation Technologies}}\\ \midrule
                Resistive Heater & $X$      & $X$   & $X$       & $X$       & $25$  & NaN & X \\
                Heatpump & $500$     & NaN & $36053$   & $2.70$    & $30$ & NaN & 1 \\
                Natural gas boiler & $60$   & NA & $1950$    & $1.00$    & $20$ & 0.90 & 1 \\ \midrule
                %\multicolumn{8}{c}{\emph{Electricity Transmission}}\\ \midrule
                Transmission & $1930$    & NaN & $36053$   & $2.70$    & $30$ & NaN & 1 \\
                \bottomrule
            \end{tabulary}

            \begin{tablenotes}
                \item SC -- supercritical, USC -- ultra-supercritical, ST -- steam turbine, GT -- gas turbine,
                CC -- combined cycle, PSP -- Pumped storage plant
                %\item [a] Hydroelectric facilities are not expanded in this model and are considered to be fully amortized.
                %\item [b] The costs for distribution infrastructure and building retrofitting are approximate (see text)
                % and they are therefore not optimised or included in the presented total system costs, but calculated
                % retrospectively and analysed in the text.
            \end{tablenotes}

        \end{threeparttable}
    \end{table*}

    Generation from non-dispatchable technologies solar photovoltaics, wind turbines and run-of-river hydro is assumed
    to follow hourly generation profiles as observed in 2016.
    German solar and wind power profiles are sourced from Open Power System Data~\citep{opsd2019}.
    For Austria, solar and wind power generation profiles were derived by dividing solar and wind power generation
    timeseries reported by Open Power System Data~\cite{opsd2019} by installed capacities published by \cite{Biermayr}.
    Similarly, generation profiles for run-of-river hydro power were derived based on generation data reported by
    \cite{ENTSOE2020}.

    Data on inflows into Austrian hydro reservoirs is not publicly available.
    However, ENTSO-E's transparency platform~\cite{ENTSOE2020} reports data on hydro storage filling levels, electricity
    generation from hydro reservoirs, and pumping by pumped hydro storages.
    Based on this data, we have approximated inflows as the part of change in (weekly) hydro storage fill levels not
    explained by (upsampled hourly) pumping or generation.
    To arrive at hourly inflows, we have interpolated our weekly estimates by piecewise cubic hermite interpolation.
    Although inflows to hydro reservoirs are given exogenous, all storage technologies (including hydro reservoir and
    pumped storage plant) are charged and dispatched endogenously.

    \subsection{Energy demand}\label{subsec:energy-demand}
    Regarding future electricity demand, government plans reveal little detail.
    According to official documents, additional renewable electricity generation of \SI{27.0}{\tera\watt\hour} will be
    required to meet electricity consummption entirely from renewable sources.\citep{Regierungsprogramm2020}.
    However, industry self generation and consumption, amounting to \SI{4.75}{\tera\watt\hour} in the base
    year~\citep{StatistikAustria2020}, is exempt~\citep{mission2030}.
    Knowing that electricity generation from renewable sources amounted to
    \SI{51.1}{\tera\watt\hour}~\citep{StatistikAustria2020}, we can infer expected total electricity consumption of
    \SI{82.85}{\tera\watt\hour} by 2030.\footnote{Balancing energy is also exempt from the renewables obligation.
    As we do not model balancing energy supply, we also do not account for this in the calculation of total demand.}
    On a side-note, expected demand in 2030 is \SI{10.45}{\tera\watt\hour} above levels observed in our base year, even
    though electrification of further sectors, most notably transport, is envisioned.
    An attempt to judge the feasibility of these plans is beyond the scope of this paper.

    For Germany, we assume electricity demand to remain at levels observed in 2016.
    In consequence, we scale reported hourly load to match annual electricity consumption published in national energy
    balances~\citep{AGEB}.
    \todo[inline]{cross-check code}

    Annual heat consumption is derived from energy balances for Austria and Germany, respectively~\citep{AGEB, OeSTAT2019}.
    Subsequently, annual heat consumption is broken down to hourly heat consumption on the basis of standard natural gas
    load profiles for space heating in the residential and commercial sectors~\citep{Almbauer2008}.
    These load profiles make use of daily average temperatures to calculate daily heat demand.
    We extract spatially resolved temperature data from ERA-5 climate data sets~\citep{era-5} and compute a
    capacity-weighted average of temperatures at locations of combined heat and power (CHP) generation units.
    Daily heat demand based on these weighted average temperatures is then broken down to hourly consumption on the
    basis of standardized factors accounting for weekday and time-of-day effects.
    Descriptive statistics of electricity and heat consumption are provided in table~\ref{tab:time-series}.

    \begin{table*}[h!t]
        \centering
        \caption{Descriptive data of time series used}\label{tab:time-series}
        \begin{tabular}{l c c c c c c l}
            \toprule
            Name & Area & Unit & mean & median & max & min & Source    \\
            \midrule
            Electricity load & AT & GW & 7.15 & 7.14 & 10.5 & 4.21 & \cite{opsd2019}      \\
            Electricity load & DE & GW & 60.2 & 59.6 & 82.8 & 34.6 & \cite{opsd2019}      \\
            District heating load & AT & GW & 2.56 & 2.43 & 5.77 & 0.91 & own clculation based on \cite{era-5, OeSTAT}\\
            District heating load & DE & GW & 14.6 & 14.1 & 26.0 & 8.9 & own calculation based on \cite{era-5, AGEB}  \\
            Wind onshore profile & AT & \%    & 0.229 & 0.142 & 0.929 & 0.000 & \cite{opsd2019}          \\
            Wind onshore profile & DE & \%    & 0.176 & 0.133 & 0.764 & 0.003 & \cite{opsd2019}          \\
            Wind offshore profile & DE & \%    & 0.371 & 0.327 & 0.900 & 0.000 & \cite{opsd2019}          \\
            Solar PV profile & AT & \%    & 0.098 & 0.012 & 0.570 & 0.000 & own calculation based on \cite{}          \\
            Solar PV profile & DE & \%    & 0.099 & 0.003 & 0.661 & 0.000 & \cite{opsd2019}          \\
            Run-of-river profile & AT & \%    & 0.564 & 0.538 & 0.971 & 0.202 & own calculation based on \cite{ENTSOE2020} \\
            Run-of-river profile & DE & \%    & 0.437 & 0.420 & 0.637 & 0.208 & \cite{ENTSOE2020} \\
            \bottomrule
        \end{tabular}
    \end{table*}


    Electricity transmission between the modelled market areas is limited to \SI{4.9}{\giga\watt}, in line with the
    introduction of a congestion management scheme by German and Austrian authorities in 2018.
    The congestion management scheme was put in place in response to so-called loop-flows from wind-rich Northern
    Germany to demand centres in Southern Germany and Austria via Germany's eastern neighbors. \todo[inline]{we would
    allow for an expansion of transmission capacities. This, however, does not happen. Shall we say we do not allow
    transmission expansion (same outcome)? Or should we say we allow for expansio, but at highly unrealistic cost, as
    the congestion scheme is artificial and removing it is a matter of a couple pieces of paper? Conduct a sensitivity run
    with 10 GW transmission capacity (which is about physical capacity in place)?}

    Monthly prices for exchange-traded fuels (hard coal, crude oil (Brent), natural gas) are retrieved from the
    International Monetary Fund's Commodity Data Portal.
    We convert these prices to \EUR/MWh based on the fuel's energy content and market exchange rates obtained from the
    European Central Bank~\citep{ECB}.
    Finally, we resample prices to hourly frequency using piecewise cubic hermite interpolation.
    Due to its low energy density lignite is not transported over large distances and consequently also not traded on
    markets.
    Instead, lignite-fired power plants are situated in proximity to lignite mines.
    According to estimates from \cite{OekoInstitut2017}, the price of lignite in Germany is close to 1.50 \EUR per
    MWh$_{th}$.
    Biomass-fired power plant run on a wide variety of solid and gaseous fuels, some of which are marketed.
    However, continued operation of biomass-fired plant typically relies on sufficient subsidies\todo{reference}.
    As a first-order approximation to complex subsidy schemes, we assume subsidized fuel cost of 6.50 \EUR per
    MWh$_{th}$.
    In consequence, biomass-fired plant are operating at or close to capacity throughout the scenarios considered.
    \footnote{Subsidies are not included in system cost, as the required level is uncertain.
    The total sum of subsidies is, however, constant across scenarios as biomass capacity can not be expanded.}

    \begin{table*}[h!t]
        \centering
        \caption{Descriptive data of price time series}
        \begin{tabular}{l c r r r r l}
            \toprule
            Name & Unit & mean & median & max & min & Source                  \\
            \midrule
            Lignite & \EUR/MWh & 1.50 & 1.50 1.50 & 1.50 & \cite{OekoInstitut2017} \\
            Coal & \EUR/MWh & 8.58 & 8.13 & 11.90 & 6.69 & \cite{IMF}              \\
            Natural gas & \EUR/MWh & 13.62 & 12.87 & 20.18 & 12.04 & \cite{IMF}              \\
            Mineral oil & \EUR/MWh & 26.40 & 26.63 & 33.44 & 18.34 & \cite{IMF}              \\
            Biomass & \EUR/MWh & 6.50 & 6.50 & 6.50 & 6.50 & own assumption          \\
            \bottomrule
        \end{tabular}
    \end{table*}

    The assumed carbon intensity of fossil fuels is displayed in Table~\ref{table:carbon-intensity}, and is based on an
    analysis by the German environmental agency~\citep{Juhrich2016}.

    \begin{table}[h!t]
        \centering
        \caption{\COO intensity of fossil fuels in tonnes \COO per MWh}
        \label{table:carbon-intensity}
        \begin{tabular}{c c c l}
            \toprule
            Lignite & Coal & Natural Gas & Mineral Oil   \\
            \midrule
            0.399 & 0.337 & 0.201 & 0.266         \\
            \bottomrule
        \end{tabular}
    \end{table}

    All data retrieval and processing scripts are available at medea's github repository
    \url{https://github.com/inwe-boku/medea} under an open MIT license.


    \section{Results and Discussion} \label{sec:results-discussion}
    To derive the social opportunity cost of not using wind power (but solar photovoltaics instead), we start by
    determining the long-run power market equilibrium without constraints on renewable capacity addition.
    This gives us our baseline scenario with minimal system costs.
    Starting from this baseline scenario, we evaluate further scenarios in which we gradually tighten the upper limit
    on wind power expansion, for example due to social conflict around wind turbine impacts on landscapes.
    We continue to tighten upper limits on wind power expansion till no wind power is allowed anymore.
    For each step, we observe system cost, total \COO emissions and other variables of interest.
    By relating the change in generated wind energy to the change in system cost (or other variables of interest), we
    derive a measure of the (approximate) marginal impact of wind power expansion (or contraction).

    \subsection{Baseline} \label{subsec:baseline}
    As a point of reference for our analysis, we determine the least cost power system configuration when addition of
    wind turbines and solar PV is unconstrained.
    Given the policy goal of generating \SI{21}{\tera\watt\hour} of electricity either from wind or solar power (with
    the remainder coming from hydro power and biomass), required capacity expansion amounts to \SI{10.4}{\giga\watt}
    wind power or \SI{24.1}{\giga\watt} of solar power if the policy is to be fulfilled with one technology only.

    In the baseline scenario, we assume capacity-weighted average capital cost of solar PV of $630$ \EUR per kW
    installed.
    This corresponds to about 56\% of PV installations being mounted on roof tops, while the remainder (44\%) is
    realized as open-space solar PV at utility scale.
    In section~\ref{subsec:sensitivity} we analyze the sensitivity of our results with respect to capital cost of solar
    PV\@.
    Moreover, we assume efficient pricing of \COO emissions, i.e.\ all otherwise external cost of \COO emissions are
    internalized through \COO pricing.
    In the baseline scenario, we presume \COO prices as low as $30$ \EUR per tonne, as there is no incentive to add
    carbon-free generation capacities above the policy target.
    As future \COO prices are uncertain but do impact renewables deployment ~\citep{Brown2020, Kirchner2019}, we
    extend the analysis to \COO prices of $60$, $90$, and $120$ \EUR per tonne.

    Given these baseline assumptions, the cost-minimizing system set-up that is consistent with policy objectives, adds
    \SI{10.4}{\giga\watt} wind power and \SI{0}{\giga\watt} solar PV to initially available capacities in Austria.
    Further, \SI{2.63}{\giga\watt} of natural gas fired capacity are added, while \SI{0.42}{\giga\watt} are decommissioned.
    \SI{0.08}{\giga\watt} off oil-fired capacity are decommissioned.
    In effect, \SI{6.23}{\giga\watt} of fossil thermal generation capacities remain active, giving rise to
    \SI{10.5}{\mega\tonne} of \COO emissions per year.
    As Austria has ample hydro storage capacity in place, no further storage (e.g. from batteries) is added.
    However, the addition of \SI{1.23}{\giga\watt} of compression heat pumps allows to use electricity for heat
    generation and to limit curtailment to \SI{1.5}{\tera\watt\hour\per\year}.
    Annual system cost in Austria amount to $2.9$ billon \EUR in this scenario.
    Net earnings from exports total $0.7$ billion \EUR, so that net system cost amounts to $2.2$ billion \EUR.

    \subsection{The opportunity cost of wind power in Austria}\label{subsec:opportunity-cost-wind}
    %Effect of \COO prices on opportunity cost of wind power:
    %- lower \COO price means less 'voluntary' RET installed
    %- more PV means more ancillary capacities installed (?)
    %- more PV means more fuel used in dispatchable plant
    %- more PV means more curtailment (?)
    %- more PV means lower exports -- large dependency on \COO price. At price of 60-120, exports fall as less wind.
    %For low \COO price exports increase as less wind. -->> Need to look at capacities in Germany!
    %- more PV means
    %* PV capacity at full wind substitution is (almost) invariant to \COO price and comes in at 24.11-24.26 GW
    %* wind onshore capacity is sensitive to \COO price. At 120 \EUR \COO, 17.07 GW wind on added, at 0 \EUR \COO10.43 GW added
    %* absolutely no heatpump added. why?
    %
    %+ No bio added in Germany (proly restriction)
    %+ Germany adds PV and wind only with \COO price at 120
    With a gradual restriction of wind power potentials, the system must increasingly rely on solar PV to generate a
    sufficient amount of electricity from renewable sources.
    Consequently, each \SI{}{\giga\watt} of wind power not installed needs to be replaced by \SI{2.35}{\giga\watt} of
    solar PV\@.
    As generation profiles of wind and solar power differ, the substitution of wind power with solar PV brings about
    changes in system operation, investment, and, consequently, system cost.

    At baseline \COO prices of $30$ \EUR per tonne,

    At higher \COO prices, wind power installation in Austria increases, as low emission generation is valued higher
    and because of the favorable complementarity of Austrian wind to (northern) German wind.


    As the admissible addition of wind power is reduced, electricity exports to Germany are being scaled back.
    This is reinforced by the similarities in generation profiles of solar PV across both countries.

    Due to the concentration of solar PV generation in time (both on a daily and on a seasonal scale), installing solar
    PV instead of wind power leads to an increase in dispatchable thermal generation.
    As thermal generation in Austria is predominantly fossil thermal, this induces an increase in \COO emissions due
    to the increased use of PV\@.

    Curtailment in Austria is declining up to

    \begin{figure*}
        \centering
        \includegraphics[width=0.8 \textwidth]{./figures/sysops.pdf}
        \caption{Opportunity cost of wind power assuming PV overnight cost of 630 EUR/kWp}
        \label{figure:system-operation-base}
    \end{figure*}

    \todo[inline]{describe changes in system configuration}

    \todo[inline]{describe marginal change in system cost due to less wind deployed}

    \todo[inline]{insert table with key scenario results}
    \begin{table}
        \caption{Key scenario results}
        \begin{tabulary}{\textwidth}{LLC}
            \toprule
            column & column & column \\
\midrule
AAA & BBB & CCC \\
\bottomrule
\end{tabulary}
\end{table}

\begin{figure*}
\centering
\begin{subfigure}[b]{0.475\textwidth}
\centering
\includegraphics[width=\textwidth]{./figures/undisturbed_base_share.pdf}
\caption{Relative substitution}
\label{fig:rel_substitution}
\end{subfigure}
\hfill
\begin{subfigure}[b]{0.475\textwidth}
\centering
\includegraphics[width=\textwidth]{./figures/undisturbed_base.pdf}
\caption{Absolute substitution}
\label{fig::abs_substitution}
\end{subfigure}
\caption{Opportunity cost of wind power assuming PV overnight cost of 630 EUR/kWp}
\label{figure:opportunity-cost-base}
\end{figure*}

\subsection{Sensitivity Analysis} \label{subsec:sensitivity}
A major factor of influence for our results are the assumed capital cost of wind turbines and solar PV.
Thus, we complement our baseline scenario with a sensitivity analysis on investment cost.
For this purpose, we vary the investment cost of solar PV between our baseline assumption of 630 EUR/kW and 275
EUR/kW that \cite{Vartiainen2019} project for utility-scale solar PV in the year 2030.

\begin{figure*}
\centering
\includegraphics[width=0.8 \textwidth]{./figures/undisturbed_low.pdf}
\caption{Opportunity cost of wind power}
\end{figure*}

%    Further uncertainty surrounds
%    •	Storage prices
%    •	Weather / full-load hours
%    •	LOOK AT WHAT HAPPENS WHEN GERMANY BUILDS MORE PV THAN ANTICIPATED (particularly in the low pv-cost scenario)

\subsection{Limitations of the Analysis}
To little surprise, the optimal deployment of renewable energy technologies heavily depends on underlying capital
cost.
Alas, uncertainty regarding future capital cost is high.
\todo[inline]{WRITE SOMETHING ON THE IMPACT OF DIFFERENT COST ASSUMPTIONS ON TOTAL SYSTEM COST.}

Nevertheless, several findings are remarkably robust to changing capital cost.
A higher penetration level of PV leads to increasing emissions of CO2 in the least-cost systems we investigated.
Due to the strong seasonal differences in electricity generation from solar PV, systems relying heavily on this technology either need to deploy excess PV capacity to boost generation in winter (resulting in heavy curtailment in summer) or, alternatively, add additional capacity to make up for the winter shortfall in electricity generation from solar PV.
This additional capacity could be storage capacity to transfer summer excesses to winter.
However, storages would have to balance seasonal differences in this case, implying the need for very high storage capacities, which are costly to realize.
Typically, it will be cheapest to add (at least some) fossil-fueled backup plants, even at \COO prices of EUR 70 per ton.
\todo[inline]{HERE, WE COULD INVESTIGATE AT WHICH CO2-PRICE FOSSIL BACKUP VANISHES.}

Moreover, the observed increase in producer surplus in line with higher CO2 penetration is robust to capital cost uncertainty.
Higher levels of PV penetration require more efforts for system integration.
\todo[inline]{THIS CAN'T BE THE FULL EXPLANATION AS PROD SURPLUS DECREASES AT VERY HIGH LEVELS OF PV}


\section{Conclusions and Policy Implication} \label{sec:conclusions-policy-implication}
We have analyzed the social willingness-to-pay for wind power (versus solar PV).

By combining estimates of the social willingness-to-pay for wind power with (spatially resolved) estimates of the negative externalities of wind turbines, we can determine the socially optimal deployment of wind turbines in Austria.
This opens up several possibilities for evidence-based policy making.

First, these estimates can inform compensation schemes for residents accepting wind turbines nearby.
Compensation might be an economically efficient measure to increase wind turbines acceptance.

Second, energy policies could be evaluated based on their impact on the opportunity cost of wind turbines.
This would allow to include otherwise unaccounted negative externalities of wind turbines in decision making, for example through a direct measure of opportunity cost, or an indirect measure of avoided wind turbines in social optimum.

Finally, our analysis also sheds some light on the distributional consequences of energy system design.


\section{Data Availability}\label{sec:data-availability}
Data and model is avaliable on github \url{}. Raw data is sourced from danish Energy Agency \url{}, ENTSO-E's
transparency database \url{},

\newpage
\bibliography{asparagus}
\bibliographystyle{elsarticle-harv}

\newpage
\appendix


\section{Description of the power system model \emph{medea}} \label{sec:medea}

\subsection{Sets} \label{sets}
Sets are denoted by upper-case latin letters, while set elements are denoted by lower-case latin letters.

\begin{table}
\caption{Sets}
\begin{tabulary}{\textwidth}{LLLp{6cm}}
\toprule
mathematical symbol & programming symbol & description & elements                                                      \\
\midrule
$f \in F$             & \texttt{f}         & fuels & \texttt{nuclear, lignite, coal, gas, oil, biomass, power}     \\
$i \in I$             & \texttt{i}         & power generation technologies & \texttt{nuc, lig\_stm, lig\_stm\_chp, lig\_boa, lig\_boa\_chp,
coal\_sub, coal\_sub\_chp, coal\_sc, coal\_sc\_chp, coal\_usc,
coal\_usc\_chp, coal\_igcc, ng\_stm, ng\_stm\_chp, ng\_ctb\_lo,
ng\_ctb\_lo\_chp, ng\_ctb\_hi, ng\_ctb\_hi\_chp, ng\_cc\_lo,
ng\_cc\_lo\_chp, ng\_cc\_hi, ng\_cc\_hi\_chp, ng\_mtr,
ng\_mtr\_chp, ng\_boiler\_chp, oil\_stm, oil\_stm\_chp,
oil\_ctb, oil\_ctb\_chp, oil\_cc, oil\_cc\_chp, bio,
bio\_chp, heatpump\_pth}                                        \\
$h \in H \subset I$     & \texttt{h(i)}    & power to heat technologies & \texttt{heatpump\_pth}                                        \\
$j \in J \subset I$     & \texttt{j(i)}    & CHP technologies & \texttt{lig\_stm\_chp, lig\_boa\_chp, coal\_sub\_chp,
            coal\_sc\_chp, coal\_usc\_chp, ng\_stm\_chp, ng\_ctb\_lo\_chp,
            ng\_ctb\_hi\_chp, ng\_cc\_lo\_chp, ng\_cc\_hi\_chp,
            ng\_mtr\_chp, ng\_boiler\_chp,
            oil\_stm\_chp, oil\_ctb\_chp, oil\_cc\_chp, bio\_chp}           \\
            $k \in K$               & \texttt{k}       & storage technologies & \texttt{res\_day, res\_week, res\_season, psp\_day,
            psp\_week, psp\_season, battery}                                \\
            $l \in L$               & \texttt{l}       & feasible operation region limits & \texttt{l1, l2, l3, l4}                                       \\
            $m \in M$               & \texttt{m}       & energy products & \texttt{el, ht}                                               \\
            $n \in N$               & \texttt{n}       & intermittent generators & \texttt{wind\_on, wind\_off, pv, ror}                         \\
            $t \in T$               & \texttt{t}       & time periods (hours)               & \texttt{t1, t2, \ldots, t8760}                                \\
            $z \in Z$               & \texttt{z}       & market zones & \texttt{AT, DE}                                               \\
            \bottomrule
        \end{tabulary}
    \end{table}

    \newpage

    \subsection{Parameters} \label{parameters}
    Parameters are denoted either by lower-case greek letters or by upper-case latin letters.
    \begin{table}
        \caption{Parameters}
        \begin{tabulary}{\textwidth}{LLLL}
            %\begin{longtable}{p{4.1cm} c l c}
            \toprule
            mathematical symbol & programming symbol & description & unit                      \\
            %\textbf{name} & \makecell[l]{\textbf{math} \\ \textbf{symbol}} & \makecell[l]{\textbf{GAMS} \\\textbf{symbol}} & \textbf{unit} \\
            % LOWER-CASE GREEK LETTERS
            \midrule
            $\delta_{z,zz}$               & \texttt{DISTANCE(z,zz)}                           & distance between countries' center of gravity & km                        \\
            $\varepsilon_{f}$             & \texttt{CO2\_INTENSITY(f)}                        & fuel emission intensity & $\text{t}_{\COO}$ / MWh   \\
            $\eta_{i,m,f}$                & \texttt{EFFICIENCY\_G(i,m,f)}                     & power plant efficiency & MWh / MWh                 \\
            $\eta^{out}_{z,k}$            & \makecell[l]{\texttt{EFFICIENCY\_S\_OUT(k)}}      & discharging efficiency &                           \\
            $\eta^{in}_{z,k}$             & \makecell[l]{\texttt{EFFICIENCY\_S\_IN(k)}}       & charging efficiency &                           \\
            $\lambda_{z}$                 & \texttt{LAMBDA(z)}                                & scaling factor for peak load &                           \\
            $\mu_{z}$                     & \texttt{VALUE\_NSE(z)}                            & value of lost load & \EUR / MWh                \\
            $\rho_{z,t,k}$                & \makecell[l]{\texttt{INFLOWS(z,t,k)}}             & inflows to storage reservoirs & MW                        \\
            $\sigma_{z}$                  & \texttt{SIGMA(z)}                                 & scaling factor for peak intermittent generation &                           \\
            $\phi_{z,t,n}$                & \texttt{GEN\_PROFILE(z,t,n)}                      & intermittent generation profile & $[0,1]$                   \\
            $\widehat{\phi}_{z,n}$        & \texttt{PEAK\_PROFILE(z,n)}                       & peak intermittent generation profile & $[0,1]$                   \\
            $\chi_{i,l,f}$                & \texttt{FEASIBLE\_INPUT(i,l,f)}                   & inputs of feasible operating region & $[0,1]$                   \\
            $\psi_{i,l,m}$                & \texttt{FEASIBLE\_OUTPUT(i,l,m)}                  & output tuples of feasible operating region & $[0,1]$                   \\
            % UPPER-CASE LATIN LETTERS
            $C^{r}_{z,n}$                 & \texttt{CAPITALCOST\_R(z,n)}                      & capital cost of intermittent generators (specific, annuity)     & \EUR / MW                 \\
            $C^{g}_{z,i}$                 & \texttt{CAPITALCOST\_G(z,i)}                      & capital cost of thermal generators (specific, annuity)          & \EUR / MW                 \\
            $C^{s}_{z,k}$                 & \makecell[l]{\texttt{CAPITALCOST\_S(z,k)}}        & capital cost of storages - power (specific, annuity)            & \EUR / MW                 \\
            $C^{v}_{z,k}$                 & \makecell[l]{\texttt{CAPITALCOST\_V(z,k)}}        & capital cost of storages - energy (specific, annuity)           & \EUR / MW                 \\
            $C^{x}$                       & \texttt{CAPITALCOST\_X}                           & capital cost of transmission capacity (specific, annuity)       & \EUR / MW                 \\
            $D_{z,t,m}$                   & \texttt{DEMAND(z,t,m)}                            & energy demand & GW                        \\
            $\widehat{D}_{z,m}$           & \texttt{PEAK\_LOAD(z,m)}                          & peak demand & GW                        \\
            $\widetilde{G}_{z,i}$         & \texttt{INITIAL\_CAP\_G(z,tec)}                   & initial capacity of dispatchable generators & GW                        \\
            $O^{g}_{i}$                   & \texttt{OM\_COST\_G\_VAR(i)}                      & variable O\&M cost of dispatchable generators & \EUR / MWh                \\
            $O^{r}_{z,n}$                 & \texttt{OM\_COST\_R\_VAR(z,n)}                    & variable O\&M cost of intermittent generators & \EUR / MWh                \\
            $P^{e}_{t,z}$                 & \texttt{PRICE\_CO2(t,z)}                          & \COO price & \EUR / $\text{t}_{\COO}$  \\
            $P_{t,z,f}$                   & \texttt{PRICE\_FUEL(t,z,f)}                       & fuel price & \EUR / MWh                \\
            $Q^{g}_{i}$                   & \texttt{OM\_COST\_G\_QFIX(i)}                     & quasi-fixed O\&M cost of dispatchable generators & \EUR / MW                 \\
            $Q^{r}_{z,n}$                 & \texttt{OM\_COST\_R\_QFIX(z,n)}                   & quasi-fixed O\&M cost of intermittent generators & \EUR / MW                 \\
            $\widetilde{R}_{z,n}$         & \texttt{INITIAL\_CAP\_R(z,n)}                     & initial capacity of intermittent generators & GW                        \\
            $\widetilde{S}^{out}_{z,k}$   & \makecell[l]{\texttt{INITIAL\_CAP\_S\_OUT(z,k)}}  & initial discharging capacity of storages & GW                        \\
            $\widetilde{S}^{in}_{z,k}$    & \makecell[l]{\texttt{INITIAL\_CAP\_S\_IN(z,k)}}   & initial charging capacity of storages & GW                        \\
            $\widetilde{V}_{z,k}$         & \makecell[l]{\texttt{INITIAL\_CAP\_V(z,k)}}       & initial energy storage capacity &                           \\
            $\widetilde{X}_{z,zz}$        & \texttt{INITIAL\_CAP\_X(z,zz)}                    & initial transmission capacity & GW                        \\
            \bottomrule
        \end{tabulary}
    \end{table}

    \newpage

    \subsection{Variables} \label{variables}
    Variables are denoted by lower-case latin letters.

    \begin{table*}
        \centering
        \caption{Variables}
        \begin{tabulary}{\textwidth}{LLLL}
            %\caption{Variables}\label{variables} \\
            \toprule
            mathematical symbol & programming symbol & description & unit      \\
            \midrule
            $b_{z,t,i,f}$                 & \texttt{b(z,t,i,f)}           & fuel burn for energy generation & GW        \\
            $c$                           & \texttt{cost\_system}         & total system cost & k\EUR     \\
            $c_{z}$                       & \texttt{cost\_zonal(z)}       & zonal system cost & k\EUR     \\
            $c^{b}_{z,t,i}$               & \texttt{cost\_fuel(z,t,i)}    & fuel cost & k\EUR     \\
            $c^{e}_{z,t,i}$               & \texttt{cost\_co2(z,t,i)}     & emission cost & k\EUR     \\
            $c^{om}_{z,i}$                & \texttt{cost\_om\_g(z,i)}     & total o\&m cost of dispatchable generators & k\EUR     \\
            $c^{om}_{z,n}$                & \texttt{cost\_om\_r(z,n)}     & total o\&m cost of intermittent generators & k\EUR     \\
            $c^{g}_{z}$                   & \texttt{cost\_invest\_g(z)}   & capital cost of generators & k\EUR     \\
            $c^{q}_{z}$                   & \texttt{cost\_nse(z)}         & total cost of non-served load & k\EUR     \\
            $c^{r}_{z}$                   & \texttt{cost\_invest\_r(z)}   & capital cost of intermittent generators & k\EUR     \\
            $c^{s,v}_{z}$                 & \texttt{cost\_invest\_sv(z)}  & capital cost of storages & k\EUR     \\
            $c^{x}_{z}$                   & \texttt{cost\_invest\_x(z)}   & capital cost of interconnectors & k\EUR     \\
            $e_{z}$                       & \texttt{emission\_co2(z)}     & \COO emissions & t \COO    \\
            $\widetilde{g}^{+}_{z,i}$     & \texttt{add\_g(z,i)}          & added capacity of dispatchables & GW        \\
            $\widetilde{g}^{-}_{z,i}$     & \texttt{deco\_g(z,i)}         & decommissioned capacity of dispatchables & GW        \\
            $g_{z,t,i,m,f}$               & \texttt{g(z,t,i,m,f)}         & energy generated by conventionals & GW        \\
            $q^{+}_{z,t}$                 & \texttt{q\_curtail(z,t)}      & curtailed energy & GW        \\
            $q^{-}_{z,t,m}$               & \texttt{q\_nse(z,t,m)}        & non-served energy & GW        \\
            $\widetilde{r}^{+}_{z,n}$     & \texttt{add\_r(z,n)}          & added capacity of intermittents & GW        \\
            $\widetilde{r}^{-}_{z,n}$     & \texttt{deco\_r(z,n)}         & decommissioned capacity of intermittents & GW        \\
            $r_{z,t,n}$                   & \texttt{r(z,t,n)}             & electricity generated by intermittents & GW        \\
            $\widetilde{s}^{+}_{z,k}$     & \texttt{add\_s(z,k)}          & added storage capacity (power)              & GW        \\
            $s^{in}_{z,t,k}$              & \texttt{s\_in(z,t,k)}         & energy stored in & GW        \\
            $s^{out}_{z,t,k}$             & \texttt{s\_out(z,t,k)}        & energy stored out & GW        \\
            $\widetilde{v}^{+}_{z,k}$     & \texttt{add\_v(z,k)}          & added storage capacity (energy)             & GWh       \\
            $v_{z,t,k}$                   & \texttt{v(z,t,k)}             & storage energy content & GWh       \\
            $w_{z,t,i,l,f}$               & \texttt{w(z,t,i,l,f)}         & operating region weight &           \\
            $\widetilde{x}^{+}_{z,zz}$    & \texttt{add\_x(z,zz)}         & added transmission capacity & GW        \\
            $x_{z,zz,t}$                  & \texttt{x(z,zz,t)}            & electricity net export & GW        \\
            \bottomrule
        \end{tabulary}
    \end{table*}

    \newpage

    \subsection{Naming system}
    \begin{table*}
        \centering
        \begin{threeparttable}
            \caption{Naming System}
            \begin{tabulary}{\textwidth}{LLLLLL}
                \toprule
                & initial capacity\tnote{$\dagger$}    & added capacity\tnote{$\ddagger$}  & decommissioned capacity\tnote{$\ddagger$} & specific investment cost\tnote{$\dagger$} & dispatch\tnote{$\ddagger$}    \\
                \midrule
                thermal units & $\widetilde{G}_{z,i}$                 & $\widetilde{g}^{+}_{z,i}$         & $\widetilde{g}^{-}_{z,i}$                 & $C^{g}_{z,i}$                             & $g_{z,t,i,m,f}$               \\
                intermittent units & $\widetilde{R}_{z,n}$                 & $\widetilde{r}^{+}_{z,n}$         & $\widetilde{r}^{-}_{z,n}$                 & $C^{r}_{z,n}$                             & $r_{z,t,n}$                   \\
                storages (power)    & $\widetilde{S}_{z,k}$                 & $\widetilde{s}^{+}_{z,k}$         & NA & $C^{s}_{z,k}$                             & $s_{z,t,k}$                   \\ % $\widetilde{s}^{-}_{z,k}$
                storages (energy)   & $\widetilde{V}_{z,k}$                 & $\widetilde{v}^{+}_{z,k}$         & NA & $C^{v}_{z,k}$                             & NA                            \\ % $\widetilde{v}^{-}_{z,t,k}$
                transmission & $\widetilde{X}_{z,zz}$                & $\widetilde{x}^{+}_{z,zz}$        & NA & $C^{x}_{z,zz}$                            & $x_{z,zz,t}$                  \\ % $\widetilde{x}^{-}_{z,zz}$
                \bottomrule
            \end{tabulary}

            \begin{tablenotes}
                \item [$\dagger$] parameter
                \item [$\ddagger$] variable
            \end{tablenotes}
        \end{threeparttable}
    \end{table*}

    \newpage

    \subsection{Mathematical description} \label{mathmodel}

    \paragraph{Model objective}
    \emph{medea} minimizes total system cost $c$, i.e. the total cost of generating electricity and heat from technologies and capacities adequate to meet demand, over a large number of decision variables, essentially representing investment and dispatch decisions in each market zone $z$ of the modelled energy systems.
    \begin{align}
        \min c = \sum_{z} (c_{z})
    \end{align}
    Zonal system costs $c_{z}$ are the sum of fuel cost $c^{b}_{z,t,i}$, emission cost $c^{e}_{z,t,i}$, operation and maintenance cost, capital costs of investment in conventional and intermittent generation ($c^{g}_{z}$, $c^{r}_{z}$), storage ($c^{s,v}_{z}$) and transmission ($c^{x}_{z}$) equipment, and the cost of non-served load ($c^{q}_{z}$) that accrues when demand is not met, e.g. when there is a power outage.
    \begin{align}
%\begin{split}
        c_{z} = \sum_{t,i}  c^{b}_{z,t,i} + \sum_{t,i} c^{e}_{z,t,i} + \sum_{i} c^{om}_{z,i} + \sum_{n} c^{om}_{z,n} + c^{g}_{z} +
        c^{r}_{z} + c^{s,v}_{z} + c^{x}_{z} + c^{q}_{z} \qquad \qquad \forall z
%\end{split}
    \end{align}
    The components of zonal system costs are calculated as given in equations (\ref{fuel_cost}) to (\ref{lost_load_cost}).
    Lower-case $c$ represent total cost, while upper-case $C$ denotes specific, annualized capital cost of technology investment.
    Prices for fuels and \COO are denoted by $P$.
    \begin{align}
        &c^{b}_{z,t,i}& =&\ \sum_{f} \left( P_{t,z,f} \: b_{t,z,i,f} \right) \qquad \qquad &\forall z,t,i \label{fuel_cost} \\
        &c^{e}_{z,t,i}& =&\ \sum_{f} \left( P^{e}_{t,z} \: e_{z,t,i} \right) \qquad \qquad &\forall z,t,i\\
        &c^{om}_{z,i}& =&\ Q^{g}_{i} \left(\widetilde{G}_{z,i} - \widetilde{g}^{-}_{z,i} + \widetilde{g}^{+}_{z,i}\right) + \sum_{t} \sum_{m} \sum_{f} \left(O^{g}_{i} \: g_{z,t,i,m,f}\right) \qquad \qquad &\forall z,i \\
        &c^{om}_{z,n}& =&\ Q^{r}_{n} \left(\widetilde{R}_{z,n} - \widetilde{r}^{-}_{z,n} + \widetilde{r}^{+}_{z,n}\right) + \sum_{t} \left(O^{r}_{n} \: r_{z,t,n}\right) \qquad \qquad &\forall z,n \\
        &c^{g}_{z}& =&\ \sum_{i} \left( C^{g}_{z,i} \: \widetilde{g}^{+}_{z,i} \right) \qquad \qquad &\forall z\\
        &c^{r}_{z}& =&\ \sum_{n} \left( C^{r}_{z,n} \: \widetilde{r}^{+}_{z,n} \right) \qquad \qquad &\forall z\\
        &c^{s,v}_{z}& =&\ \sum_{k} \left( C^{s}_{z,k} \: \widetilde{s}^{+}_{z,k} + C^{v}_{z,k} \: v^{+}_{z,k} \right) \qquad \qquad &\forall z\\
        &c^{x}_{z}& =&\ \frac{1}{2} \: \sum_{zz} (C^{x} \: \delta_{z,zz} \: \widetilde{x}^{+}_{z,zz}) \qquad \qquad &\forall z \label{transmission_expansion_cost}\\
        &c^{q}_{z}& =&\ \mu \sum_{t} \sum_{m} q^{-}_{z,t,m} \qquad \qquad &\forall z \label{lost_load_cost}
    \end{align}

    \paragraph{Market clearing}
    In each hour, the markets for electricity and heat have to clear.
    Equation (\ref{market_clearing_el}) ensures that the total supply from conventional and intermittent sources, and storages equals total electricity demand plus net exports, electricity stored and used for heat generation.
    Likewise, equation (\ref{market_clearing_ht}) clears the heat market by equating heat generation to heat demand.
    \begin{align}
        \begin{split}
            \sum_{i} \sum_{f} g_{z,t,i,\text{el},f} + \sum_{k} s^{out}_{z,t,k} + \sum_{n} r_{z,t,n} &=  \\ D_{z,t,\text{el}} + \sum_{i} b_{z,t,i,\text{el}} + & \sum_{k} s^{in}_{z,t,k} + \sum_{zz} x_{z,zz,t} - q^{-}_{z,t,\text{el}} + q^{+}_{z,t} \qquad \forall z,t
        \end{split}
        \label{market_clearing_el}
    \end{align}
    \begin{align}
        \sum_{i} \sum_{f} g_{z,t,i,\text{ht},f} = D_{z,t,\text{ht}} - q^{-}_{z,t,\text{ht}} \qquad \forall z,t \label{market_clearing_ht}
    \end{align}
    \emph{medea} can be thought of as representing energy-only electricity and heat markets without capacity payments. Then, the marginals of the market clearing equations (\ref{market_clearing_el}) and (\ref{market_clearing_ht}), $\partial C / \partial D_{z,t,m}$, can be interpreted as the zonal prices for electricity and heat, respectively.

    \paragraph{Energy generation}
    Energy generation $g_{z,t,i,m,f} \geq 0$ is constrained by available installed capacity, which can be adjusted through investment ($\widetilde{g}^{+}_{z,i} \geq 0$) and decommissioning $\widetilde{g}^{-}_{z,i} \geq 0$.
    \begin{align}
        \sum_{f} g_{z,t,i,m,f} \leq \widetilde{G}_{z,i} + \widetilde{g}^{+}_{z,i} - \widetilde{g}^{-}_{z,i} \qquad \qquad \forall z,t,i,m
    \end{align}
    Generator efficiency $\eta$ determines the amount of fuel $b_{z,t,i,f} \geq 0$ that needs to be spent in order to generate a given amount of energy.
    \begin{align}
        g_{z,t,i,m,f} = \sum_{f} \eta_{i,m,f} \: b_{z,t,i,f} \qquad \qquad \forall z,t,i \notin J, f
    \end{align}

    \paragraph{Thermal co-generation}
    Co-generation units jointly generate heat and electricity. All feasible combinations of heat and electricity generation along with the corresponding fuel requirement are reflected in so-called `feasible operating regions'.
    The elements $l \in L$ span up a three-dimensional, convex feasible operating region for each co-generation technology.
    The weights $w_{z,t,i,l,f} \geq 0$ form a convex combination of the corners $l$, which are scaled to the available installed capacity of each co-generation technology.
    Defining weights over fuels allows co-generation units to switch fuels between multiple alternatives.
    Heat and electricity output along with the corresponding fuel requirement is then set according to the chosen weights.
    \begin{align}
        \sum_{l} \sum_{f} w_{z,t,i,l,f} = \widetilde{G}_{z,i} + \widetilde{g}^{+}_{z,i} - \widetilde{g}^{-}_{z,i} \qquad \qquad \forall z,t,i \in J \\
        g_{z,t,i,m,f} = \sum_{l} \sum_{f} \psi_{i,l,m} \: w_{z,t,i,l,f} \qquad \qquad \forall z,t,i \in J, m \\
        b_{z,t,i,f} = \sum_{l} \chi_{i,l,f} \: w_{z,t,i,l,f} \qquad \qquad \forall z,t,i \in J, f\\
        w(z,t,i,l,f) = 0 \qquad \qquad \forall z,t,i,k,f: \chi_{i,l,f} = 0
    \end{align}

    \paragraph{Intermittent electricity generation}
    Electricity generation from intermittent sources wind (on-shore and off-shore), solar irradiation, and river runoff follows generation profiles $\phi_{z,t,n} \in [0,1]$ and is scaled according to corresponding installed ($ \widetilde{R}_{z,n}$) and added ($\widetilde{r}^{+}_{z,n} \geq 0$) capacity.
    \begin{align}
        r_{z,t,n} = \phi_{z,t,n} \: \left( \widetilde{R}_{z,n} - \widetilde{r}^{-}_{z,n} + \widetilde{r}^{+}_{z,n} \right) \qquad \qquad \forall z,t,n
    \end{align}

    \paragraph{Electricity storages}
    Charging ($s^{in}_{z,t,k} \geq 0$) and discharging ($s^{out}_{z,t,k} \geq 0$) of storages is constrained by the storages' installed ($\widetilde{S}^{in}_{z,k}, \widetilde{S}^{out}_{z,k}$) and added ($\widetilde{s}^{+}_{z,k} \geq 0$) charging and discharging power, respectively. Similarly, the total energy that can be stored is constrained by the storage technology's initial ($\widetilde{V}_{z,k}$) and added ($\widetilde{v}^{+}_{z,k} \geq 0$) energy capacity.
    \begin{align}
        s^{out}_{z,t,k} \leq \widetilde{S}^{out}_{z,k} + \widetilde{s}^{+}_{z,k} \qquad \qquad \forall z,t,k \\
        s^{in}_{z,t,k} \leq \widetilde{S}^{in}_{z,k} + \widetilde{s}^{+}_{z,k} \qquad \qquad \forall z,t,k \\
        v_{z,t,k} \leq \widetilde{V}_{z,k} + \widetilde{v}^{+}_{z,k} \qquad \qquad \forall z,t,k
    \end{align}
    Storage operation is subject to a storage balance, such that the current energy content must be equal to the previous period's energy content plus all energy flowing into the storage less all energy flowing out of the storage.
    \begin{align}
        v_{z,t,k} = \rho_{z,t,k} + \eta^{in}_{z,k} \: s^{in}_{z,t,k} - (\eta^{out}_{z,k})^{-1} \: s^{out}_{z,t,k} + v_{z,t-1,k} \qquad \qquad \forall z,t,k: t>1, \: \eta^{out}_{z,k} > 0
    \end{align}
    Since the model can add storage power capacity and energy capacity independently, we require a storage to hold at least as much energy as it could store in (or out) in one hour.
    \begin{align}
        \widetilde{v}^{+}_{z,k} \geq \widetilde{s}^{+}_{z,k} \qquad \qquad \forall z,k
    \end{align}

    \paragraph{Emission accounting}
    Burning fossil fuels for energy generation produces emissions of carbon dioxide (\COO). The amount of \COO emitted is
    tracked by the following equation
    \begin{align}
        e_{z,t,i} = \sum_{f} \left( \varepsilon_{f} \: b_{z,t,i,f} \right) \qquad \qquad \forall z,t,i
    \end{align}

    \paragraph{Electricity exchange}
    Implicitly, \emph{medea} assumes that there are no transmission constraints within market zones.
    However, electricity exchange between market zones is subject to several constraints.

    First, exchange between market zones is constrained by available transfer capacities. Transfer capacities can be expanded at constant, specific investment cost (see equation (\ref{transmission_expansion_cost})). This rules out economies of scale in transmission investment that might arise in interconnected, meshed grids.
    \begin{align}
        x_{z,zz,t} \leq \widetilde{X}_{z,zz} + \widetilde{x}^{+}_{z,zz} \qquad \qquad \forall z, zz, t \\
        x_{z,zz,t} \geq - \left( \widetilde{X}_{z,zz} + \widetilde{x}^{+}_{z,zz} \right) \qquad \qquad \forall z, zz, t
    \end{align}
    By definition, electricity net exports $x_{z,zz,t}$ from $z$ to $zz$ must equal electricity net imports of $zz$ from $z$.
    \begin{align}
        x_{z,zz,t} = -x_{zz,z,t} \qquad \qquad \forall z, zz, t
    \end{align}
    Added transmission capacities can be used in either direction.
    \begin{align}
        \widetilde{x}^{+}_{z,zz} = \widetilde{x}^{+}_{zz,z} \qquad \qquad \forall z, zz
    \end{align}
    Finally, electricity cannot flow between zones where there is no transmission infrastructure in place (including intra-zonal flows).
    \begin{align}
        x_{z,zz,t} = 0 \qquad \qquad \forall z, zz, t: \widetilde{X}_{z,zz} = 0 \\
        x_{zz,z,t} = 0 \qquad \qquad \forall z, zz, t: \widetilde{X}_{z,zz} = 0
    \end{align}

    \paragraph{Decommissioning}
    Keeping plant available for generation gives rise to quasi-fixed operation and maintenance costs.
    Such cost can be avoided by decommissioning an energy generator. This is modelled as a reduction in generation capacity, which cannot exceed installed capacity.
    \begin{align}
        \widetilde{g}^{-}_{z,i} &\leq \widetilde{G}_{z,i} + \widetilde{g}^{+}_{z,i} \qquad \qquad \forall z,i \\
        \widetilde{r}^{-}_{z,n} &\leq \widetilde{R}_{z,n} + \widetilde{r}^{+}_{z,n} \qquad \qquad \forall z,n
    \end{align}

    \paragraph{Ancillary services}
    Power systems require various system services for secure and reliable operation, such as balancing services or voltage support through the provision of reactive power. Such system services can only be supplied by operational generators. Thus, we approximate system service provision by a requirement on the minimal amount of spinning reserves operating at each hour.
    We assume that ancillary services are supplied by conventional (thermal) power plants, hydro power plants, and storages.
    The requirement for spinning reserves is proportional to electricity peak load $\widehat{D}_{z,\text{el}} = \max_{t} D_{z,t,\text{el}}$ and peak generation from wind and solar resources, where $\widehat{\phi}_{z,n} = \max_{t} \phi_{z,t,n}$.
    \begin{align}
        \sum_{i} \sum_{f} \left( g_{z,t,i,\text{el},f} \right) + r_{z,t,\text{ror}} + \sum_{k} \left( s^{out}_{z,t,k} + s^{in}_{z,t,k} \right) \geq \lambda_{z} \widehat{D}_{z,\text{el}} + \sigma_{z} \sum_{n \setminus \{ \text{ror}\}} \widehat{\phi}_{z,n} (\widetilde{R}_{z,n} + \widetilde{r}^{+}_{z,n}) \qquad \forall z,t
    \end{align}

    \paragraph{Curtailment}
    Electricity generated from intermittent sources can be curtailed (disposed of) without any further cost (apart from implicit opportunity cost).
    \begin{align}
        q^{+}_{z,t} \leq \sum_{n} r_{z,t,n} \qquad \qquad \forall z, t
    \end{align}

\end{document}