\documentclass[11pt,a4paper]{article}
\usepackage[utf8]{inputenc}
\usepackage[fleqn]{amsmath}
\usepackage{amsfonts}
\usepackage{amssymb}
\DeclareMathOperator*{\argmax}{arg\,max}

\usepackage{natbib}
\usepackage{makecell}
\usepackage[gen]{eurosym} 
\usepackage[version=4]{mhchem}
%\usepackage{courier} 
\usepackage[inner=2.5cm,outer=2.5cm, bottom=3cm]{geometry}

\title{Description of the power system model \emph{medea}}
%\title{Mathematical description of the power system model \emph{medea}}
\date{\today}
\author{Sebastian Wehrle\thanks{Over recent years, \emph{medea} has grown into its current state thanks to inspiration and contributions from many friends and colleagues. 
In particular, I'd like to thank Johannes Schmidt, who contributed substantial parts of the initial model code, for getting \emph{medea} started and for keeping us running; Dieter Mayr and Stefan Höltinger for early contributions; Peter Regener for inspiration and endurance with all things programming. Current work on \emph{medea} is funded through grateful support from the European Research Council ("reFUEL" ERC-2017-STG 758149)} \\University of Natural Resources and Life Sciences, Vienna}

\begin{document}
\maketitle
\section{Overview}
\emph{medea} is a simple, stylized and parsimonious model of interconnected power and heating systems in Western and Central Europe.
It simulates investment in intermittent and conventional electricity and heat generation technologies as well as in cross-border electricity transmission capacities.
At the same time, the model determines the system-cost minimizing hourly dispatch of electricity and heat generators to meet price-inelastic demand.
Model results include hourly energy generation by technology and the associated fuel use and CO2 emissions, investment in and decommissioning of conventional and renewable generators and energy storages, hourly cross-border flows of electricity and potentially required transmission capacity expansion, as well as producer and consumer surplus.

A detailed description of the model is provided in the following. Section \ref{sets} gives an overview of the sets and set elements used in \emph{medea}. Sections \ref{parameters} and \ref{variables} introduce the model's parameters and variables, while section \ref{mathmodel} gives a detailed description of the model's mathematical formulation.

\newpage

\section{Sets} \label{sets}
\begin{tabular}{l l l l}
\textbf{name} & \makecell[l]{\textbf{math} \\ \textbf{symbol}} & \makecell[l]{\textbf{GAMS} \\\textbf{symbol}} & \textbf{elements} \\
\hline \hline
market zones & $z \in Z$ & \texttt{z} & \texttt{AT, DE} \\ \hline

\makecell[l]{time periods \\(hours)} & $t \in T$ & \texttt{t} & \texttt{t1, t2, \ldots, t8760} \\ \hline

\makecell[l]{power \\generation \\technologies} & $i \in I$ & \texttt{tec} & 

\makecell[l]{ \texttt{nuc, lig\_stm, lig\_stm\_chp,}\\
	\texttt{lig\_boa, lig\_boa\_chp,} \\
	\texttt{coal\_sub, coal\_sub\_chp,} \\
	\texttt{coal\_sc, coal\_sc\_chp,}\\
	\texttt{coal\_usc, coal\_usc\_chp,} \\
	\texttt{coal\_igcc,}\\
	\texttt{ng\_stm, ng\_stm\_chp,} \\
	\texttt{ng\_ctb\_lo, ng\_ctb\_lo\_chp,}\\ 
	\texttt{ng\_ctb\_hi, ng\_ctb\_hi\_chp,}\\
	\texttt{ng\_cc\_lo, ng\_cc\_lo\_chp,}\\
	\texttt{ng\_cc\_hi, ng\_cc\_hi\_chp,}\\
	\texttt{ng\_mtr, ng\_mtr\_chp,} \\
	\texttt{ng\_boiler\_chp,} \\
	\texttt{oil\_stm, oil\_stm\_chp,} \\ 
	\texttt{oil\_ctb, oil\_ctb\_chp,} \\
	\texttt{oil\_cc, oil\_cc\_chp,} \\
	\texttt{bio, bio\_chp,}
	\texttt{heatpump\_pth} 
} \\ \hline

\makecell[l]{CHP \\technologies} & $j \in J \subset I$ & \texttt{tec\_chp} & \makecell[l]{
	\texttt{lig\_stm\_chp, lig\_boa\_chp,}\\
	\texttt{coal\_sub\_chp, coal\_sc\_chp,}\\
	\texttt{coal\_usc\_chp, ng\_stm\_chp,}\\ 
	\texttt{ng\_ctb\_lo\_chp, ng\_ctb\_hi\_chp,}\\
	\texttt{ng\_cc\_lo\_chp, ng\_cc\_hi\_chp,}\\
	\texttt{ng\_mtr\_chp, ng\_boiler\_chp,}\\
	\texttt{oil\_stm\_chp, oil\_ctb\_chp,} \\ 
	\texttt{oil\_cc\_chp, bio\_chp} } \\ \hline

\makecell[l]{power to heat \\ technologies} & $h \in H \subset I$ & \texttt{tec\_pth} & \texttt{heatpump\_pth}  \\ \hline

\makecell[l]{storage \\technologies} & $k \in K$ & \texttt{tec\_strg} & \makecell[l]{
	\texttt{res\_day, res\_week, res\_season,} \\ 
	\texttt{psp\_day, psp\_week, psp\_season,} \\
	\texttt{battery}
	} \\ \hline

\makecell[l]{intermittent \\generators} & $n \in N$ & \texttt{tec\_itm} & \texttt{wind\_on, wind\_off, pv, ror}\\ \hline

fuels & $f \in F$ & \texttt{f} & \makecell[l]{\texttt{nuclear, lignite, coal,} \\ \texttt{gas, oil, biomass, power}} \\ \hline

\makecell[l]{feasible operation \\region limits}& $l \in L$ & \texttt{l} & \texttt{l1, l2, l3, l4} \\ \hline

energy products & $m \in M$ & \texttt{prd} & \texttt{el, ht} \\ \hline \hline
\end{tabular}

\section{Parameters} \label{parameters}
\begin{tabular}{l l l l}
\textbf{name} & \makecell[l]{\textbf{math} \\ \textbf{symbol}} & \makecell[l]{\textbf{GAMS} \\\textbf{symbol}} & \textbf{unit} \\
\hline \hline
\makecell[l]{minimal \\conventional \\generation} & $\sigma_{z}$ & \texttt{ANCIL\_SERVICE\_LVL(z)} & GW \\ \hline
energy demand & $D_{z,t,p}$ & \texttt{CONSUMPTION(z,t,prd)} & GW \\ \hline
\makecell[l]{power plant \\efficiency} & $\eta_{i,p,f}$ & \texttt{EFFICIENCY(tec,prd,f)} & MWh / MWh\\ \hline
\makecell[l]{fuel emission \\intensity} & $\varepsilon_{f}$ & \texttt{EMISSION\_INTENSITY(f)} & $\text{t}_{\ce{CO2}}$ / MWh\\ \hline
\makecell[l]{inputs of \\feasible operating \\region} & $\chi_{i,l,f}$ & \texttt{FEASIBLE\_INPUT(tec,l,f)} & $[0,1]$ \\ \hline
\makecell[l]{output tuples of \\feasible operating \\region} & $\psi_{i,l,p}$ & \texttt{FEASIBLE\_OUTPUT(tec,l,prd)} & $[0,1]$ \\ \hline
\makecell[l]{intermittent \\generation profile} & $\phi_{z,t,n}$ & \texttt{GEN\_PROFILE(z,t,tec\_itm)} & $[0,1]$ \\ \hline
%\makecell[l]{exports to not \\modelled regions} & $X$ & \texttt{FLOW\_EXPORTS(z,t)} & GW \\ \hline
%\makecell[l]{imports from not \\modelled regions} & $X$ & \texttt{FLOW\_IMPORTS(z,t)} & GW \\ \hline
\makecell[l]{installed capacity \\of intermittent \\generators} & $\bar{r}^{0}_{z,n}$ & \texttt{INSTALLED\_CAP\_ITM(z,tec\_itm)} & GW \\ \hline
\makecell[l]{installed capacity \\of thermal generators} & $\bar{g}^{0}_{z,i}$ & \texttt{INSTALLED\_CAP\_THERM(z,tec)} & GW \\ \hline
\makecell[l]{capital cost of \\intermittent generators \\(specific, annuity)} & $\kappa_{z,n}$ & \texttt{INVESTCOST\_ITM(z,tec\_itm)} & k\EUR / GW \\ \hline
\makecell[l]{capital cost of \\thermal generators \\(specific, annuity)} & $\kappa_{z,i}$ & \texttt{INVESTCOST\_THERMAL(z,tec)} & k\EUR / GW \\ \hline
\makecell[l]{capital cost of \\storages - power \\(specific, annuity)} & $\kappa^{P}_{z,k}$ & \makecell[l]{\texttt{STORAGE\_PROPERTIES} \\ \texttt{(z,tec\_strg,'cost\_power')}} & k\EUR / GW \\ \hline
\makecell[l]{capital cost of \\storages - energy \\(specific, annuity)} & $\kappa^{E}_{z,k}$ & \makecell[l]{\texttt{STORAGE\_PROPERTIES} \\ \texttt{(z,tec\_strg,'cost\_energy')}} & k\EUR / GW \\ \hline
\makecell[l]{installed available \\transmission capacity} & $\bar{x}^{0}_{z,zz}$ & \texttt{ATC(z,zz)} & GW \\ \hline
\makecell[l]{capital cost of \\transmission capacity} & $\kappa^{ic}_{z}$ & \texttt{INVESTCOST\_ATC} & k\EUR / GW \\ \hline
\makecell[l]{distance between \\countries} & $\delta_{z,zz}$ & \texttt{KM(z,zz)} & km \\ \hline
%\makecell[l]{count 100 MW slices \\of same technology} & $|i_{z}|$ & \texttt{NUM(z,tec)} & . \\ \hline
quasi-fixed O\&M cost & $o^{q}_{i}$ & \texttt{OM\_FIXED\_COST(tec)} & k\EUR \\ \hline
variable O\&M cost & $o^{v}_{i}$ & \texttt{OM\_VAR\_COST(tec)} & \EUR / MWh \\ \hline
\hline
\end{tabular}

\begin{tabular}{l l l l}
\textbf{name} & \makecell[l]{\textbf{math} \\ \textbf{symbol}} & \makecell[l]{\textbf{GAMS} \\\textbf{symbol}} & \textbf{unit} \\
\hline \hline
% day-ahead price & $p_{el}$ & \texttt{PRICE\_DA(t,z)} & EUR / MWh \\ \hline
\ce{CO2} price & $P^{e}_{t,z}$ & \texttt{PRICE\_EUA(t,z)} & \EUR / $\text{t}_{\ce{CO2}}$ \\ \hline
fuel price & $P_{t,z,f}$ & \texttt{PRICE\_FUEL(t,z,f)} & \EUR / MWh \\ \hline
reservoir inflows & $\rho_{z,t,k}$ & \makecell[l]{\texttt{RESERVOIR\_INFLOWS} \\ \texttt{(z,t,tec\_strg)}} & MW \\ \hline
max power out & $\bar{s}^{out}_{z,k}$ & \makecell[l]{\texttt{STORAGE\_PROPERTIES} \\ \texttt{(z,tec\_strg,'power\_out')}} & GW \\ \hline
max power in & $\bar{s}^{in}_{z,k}$ & \makecell[l]{\texttt{STORAGE\_PROPERTIES} \\ \texttt{(z,tec\_strg,'power\_in')}} & GW \\ \hline
max energy stored & $\bar{v}^{0}_{z,k}$ & \makecell[l]{\texttt{STORAGE\_PROPERTIES} \\ \texttt{(z,tec\_strg,'energy\_max')}} & . \\ \hline
efficiency power out & $\eta^{out}_{z,k}$ & \makecell[l]{\texttt{STORAGE\_PROPERTIES} \\ \texttt{(z,tec\_strg,'efficiency\_out')}} & . \\ \hline
efficiency power in & $\eta^{in}_{z,k}$ & \makecell[l]{\texttt{STORAGE\_PROPERTIES} \\ \texttt{(z,tec\_strg,'efficiency\_in')}} & . \\ \hline
value of lost load & $\mu$ & \texttt{VALUE\_NSE(z)} & \EUR / MWh \\ \hline
\hline
\end{tabular}

\section{Variables} \label{variables}
\begin{tabular}{l l l l}
\textbf{name} & \makecell[l]{\textbf{math} \\ \textbf{symbol}} & \makecell[l]{\textbf{GAMS} \\\textbf{symbol}} & \textbf{Unit} \\
\hline \hline
%cost variables
zonal system cost & $C_{z}$ & \texttt{cost(r)} & k\EUR \\ \hline
emission cost & $C^{e}_{z,t,i}$ & \texttt{cost\_emission(z,t,tec)} & k\EUR \\ \hline
fuel cost & $C^{f}_{z,t,i}$ & \texttt{cost\_fuel(z,t,tec)} & k\EUR \\ \hline
total o\&m cost & $C^{om}_{z,i}$ & \texttt{cost\_om(z,tec)} & k\EUR \\ \hline
\makecell[l]{capital cost of \\generators} & $C_{z}^{inv,i}$ & \texttt{cost\_invgen(z)} & k\EUR \\ \hline
\makecell[l]{capital cost of \\storages} & $C_{z}^{inv,k}$ & \texttt{cost\_invstrg(z)} & k\EUR \\ \hline
\makecell[l]{capital cost of \\interconnectors} & $C_{z}^{inv,ic}$ & \texttt{cost\_gridexpansion(z)} & k\EUR \\ \hline
% investment and decommissioning variables
\makecell[l]{added capacity of \\intermittents} & $\bar{r}^{+}_{z,n}$ & \texttt{invest\_res(z,tec\_itm)} & GW \\ \hline
\makecell[l]{added capacity of \\conventionals} & $\bar{g}^{+}_{z,i}$ & \texttt{invest\_thermal(z,tec)} & GW \\ \hline
\makecell[l]{added storage \\capacity (power)} & $\bar{s}^{+}_{z,k}$ & \texttt{invest\_storage\_power(z,tec\_strg)} & GW \\ \hline
\makecell[l]{added storage \\capacity (energy)} & $\bar{v}^{+}_{z,k}$ & \texttt{invest\_storage\_energy(z,tec\_strg)} & GWh \\ \hline
\makecell[l]{added transmission \\capacity} & $\bar{x}^{+}_{z,zz}$ & \texttt{invest\_atc(z,zz)} & GW \\ \hline
\makecell[l]{decommissioned \\capacity of \\conventionals} & $\bar{g}^{-}_{z,i}$ & \texttt{decommission(z,tec)} & GW \\ \hline
% dispatch variables
\makecell[l]{energy generated by \\conventionals} & $g_{z,t,i,p}$ & \texttt{q\_gen(z,t,tec,prd)} & GW \\ \hline
\makecell[l]{electricity generated \\by intermittents} & $r_{z,t,n}$ & \texttt{q\_itm(z,t,tec\_itm)} & GW \\ \hline
\makecell[l]{operating region \\weight} & $w_{z,t,i,l}$ & \texttt{cc\_weights(z,t,tec,l)} & . \\ \hline
\makecell[l]{fuel burn for energy \\generation} & $b_{z,t,i,f}$ & \texttt{q\_fueluse(z,t,tec,f)} & GW \\ \hline
energy stored in & $s^{in}_{z,t,k}$ & \texttt{q\_store\_in(z,t,tec\_strg)} & GW \\ \hline
energy stored out & $s^{out}_{z,t,k}$ & \texttt{q\_store\_out(z,t,tec\_strg)} & GW \\ \hline
\makecell[l]{storage energy \\content} & $v_{z,t,k}$ & \texttt{storage\_level(z,t,tec\_strg)} & GWh \\ \hline
\makecell[l]{electricity net \\export} & $x_{z,zz,t}$ & \texttt{flow(z,zz,t)} & GW \\ \hline
curtailed energy & $\Omega^{+}_{z,t}$ & \texttt{q\_curtail(z,t)} & GW \\ \hline
non-served energy & $\Omega^{-}_{z,t,p}$ & \texttt{q\_nonserved(z,t,prd)} & GW \\ \hline
% other variables
\ce{CO2} emissions & $e_{z}$ & \texttt{emissions(z)} & t \ce{CO2} \\ \hline
\hline
\end{tabular}

\newpage
\section{Mathematical description} \label{mathmodel}

\paragraph{Model objective}
\emph{medea} minimizes total system cost $C$, i.e. the total cost of generating electricity and heat from technologies and capacities adequate to meet demand, over a large number  of decision variables, essentially representing investment and dispatch decisions in each market zone $z$ of the modelled energy systems.
\begin{align}
\min C = \sum_{z} (C_{z})
\end{align}
Zonal system costs $C_{z}$ are the sum of fuel cost $C^{f}_{z,t,i}$, emission cost $C^{e}_{z,t,i}$, operation and maintenance cost $C^{om}_{z,i}$, capital costs of investment in conventional and intermittent generation ($C^{inv,i}_{z}$, $C^{inv,n}_{z}$), storage ($C^{inv,k}_{z}$) and transmission ($C^{inv,ic}_{z}$) equipment, and the cost of non-served load ($C^{nse}_{z}$) that accrues when demand is not met, e.g. when there is a power outage. 
\begin{align}
%\begin{split}
C_{z} = \sum_{t,i}  C^{f}_{z,t,i} + \sum_{t,i} C^{e}_{z,t,i} + \sum_{i} C^{om}_{z,i} + C^{inv,i}_{z} + 
 C^{inv,n}_{z} + C^{inv,k}_{z} + C^{inv,ic}_{z} + C^{nse}_{z} \qquad \qquad \forall z
%\end{split}
\end{align}
The components of zonal system costs are calculated as given in equations (\ref{fuel_cost}) to (\ref{lost_load_cost}).
Upper-case $C$ represent total cost, while $\kappa$ denotes specific, annualized capital cost of technology investment. Prices for fuels and \ce{CO2} are denoted by uppercase $P$.
\begin{align}
&C^{f}_{z,t,i}& =&\ \sum_{f} \left( P_{t,z,f} \: b_{t,z,i,f} \right) \qquad \qquad &\forall z,t,i \label{fuel_cost} \\
&C^{e}_{z,t,i}& =&\ \sum_{f} \left( P^{e}_{t,z} \: \varepsilon_{f} \: b_{t,z,i,f} \right) \qquad \qquad &\forall z,t,i\\
&C^{om}_{z,i}& =&\ \left(\bar{g}^{0}_{z,i} - \bar{g}^{-}_{z,i} + \bar{g}^{+}_{z,i}\right) o^{q}_{i} + \sum_{t,m} \left(o^{v}_{i} \: g_{z,t,i,m}\right) \qquad \qquad &\forall z,i \\
&C^{inv,i}_{z}& =&\ \kappa_{z,i} \: \bar{g}^{+}_{z,i} \qquad \qquad &\forall z\\
&C^{inv,n}_{z}& =&\ \kappa_{z,n} \: \bar{r}^{+}_{z,n} \qquad \qquad &\forall z\\
&C^{inv,k}_{z}& =&\ \kappa^{P}_{z,k} \: \bar{s}^{+}_{z,k} + \kappa^{E}_{z,k} \: v^{+}_{z,k} \qquad \qquad &\forall z\\
&C^{inv,ic}_{z}& =&\ \kappa^{ic}_{z} \: \delta_{z,zz} \: \bar{x}^{+}_{z,zz} \qquad \qquad &\forall z \label{transmission_expansion_cost}\\
&C^{nse}_{z}& =&\ \mu \sum_{t,m} \Omega^{-}_{z,t,m} \qquad \qquad &\forall z \label{lost_load_cost}
\end{align}
In line with \cite{citation}, we estimate the specific cost of non-served load $\mu$ at 12500 EUR per MWh.

\paragraph{Market clearing}
In each hour, the markets for electricity and heat have to clear.
Equation (\ref{market_clearing_el}) ensures that the total supply from conventional and intermittent sources, and storages equals total electricity demand plus net exports, electricity stored and used for heat generation. 
Likewise, equation (\ref{market_clearing_ht}) clears the heat market by equating heat generation to heat demand.
\begin{align}
\sum_{i} g_{z,t,i,\text{el}} + \sum_{k} s^{out}_{z,t,k} + \sum_{n} r_{z,t,n} =
D_{z,t,\text{el}} + \sum_{i} b_{z,t,i,\text{el}} + \sum_{k} s^{in}_{z,t,k} + \sum_{zz} x_{z,zz,t} 
- \Omega^{-}_{z,t,\text{el}} + \Omega^{+}_{z,t} \qquad \forall z,t \label{market_clearing_el} \\
\sum_{i} g_{z,t,i,\text{ht}} = D_{z,t,\text{ht}} - \Omega^{-}_{z,t,\text{ht}} \qquad \forall z,t \label{market_clearing_ht}
\end{align}
\emph{medea} can be thought of as representing energy-only electricity and heat markets without capacity payments. Then, the marginals of the market clearing equations (\ref{market_clearing_el}) and (\ref{market_clearing_ht}), $\partial C / \partial D_{z,t,m}$, can be interpreted as the zonal prices for electricity and heat, respectively.

\paragraph{Thermal electricity generation}
Energy generation is constrained by available installed capacity, which can be adjusted through investment and decommissioning.
\begin{align}
g_{z,t,i,m} \leq \argmax_{l}(\psi_{i,l,m}) \left( \bar{g}^{0}_{z,i} + \bar{g}^{+}_{z,i} - \bar{g}^{-}_{z,i} \right) \qquad \qquad \forall z,t,i,m
\end{align}
Generator efficiency $\eta$ determines the amount of fuel that needs to be burnt in order to generate a given amount of electricity.
\begin{align}
g_{z,t,i,\text{el}} = \sum_{f} \eta_{i,\text{el},f} \: b_{z,t,i,f} \qquad \qquad \forall z,t,i \notin J
\end{align}

\paragraph{Electric heat generation}
Heat pumps and electric boilers use electricity to generate heat. The amount of electricity required to generate a given amount of heat depends on the unit's efficiency $\eta$. For heat pumps, the annual COP $\eta$ is assumed to be $3$.
\begin{align}
g_{z,t,i,\text{ht}} = \sum_{f} \eta_{i,\text{ht},f} \: b_{z,t,i,f} \qquad \qquad \forall z,t,i \in H
\end{align}


\paragraph{Thermal co-generation}
Co-generation units jointly generate heat and electricity. All feasible combinations of heat and electricity generation along with the corresponding fuel requirement are reflected in so-called `feasible operating regions'. 
The elements $l \in L$ span up a three-dimensional, convex feasible operating region for each co-generation technology. 
The weights $w$ form a convex combination of the corners $l$, which are scaled to the available installed capacity of each co-generation technology.
Heat and electricity output along with the corresponding fuel requirement is then set according to the chosen weights.
\begin{align}
\sum_{l} w_{z,t,i,l} = \bar{g}^{0}_{z,i} + \bar{g}^{+}_{z,i} - \bar{g}^{-}_{z,i} \qquad \qquad \forall z,t,i \in J \\
g_{z,t,i,m} \leq \sum_{l} w_{z,t,i,l} \: \psi_{i,l,m} \qquad \qquad \forall z,t,i \in J, m \\
b_{z,t,i,f} \geq \sum_{l} w_{z,t,i,l} \: \chi_{i,l,f} \qquad \qquad \forall z,t,i \in J, f
\end{align}

\paragraph{Intermittent electricity generation}
Electricity generation from intermittent sources wind (on-shore and off-shore), solar irradiation, and river runoff follows generation profiles $\phi_{z,t,n} \in [0,1]$ and is scaled according to corresponding installed and added capacity.
\begin{align}
r_{z,t,n} = \phi_{z,t,n} \: \left( \bar{r}^{0}_{z,n} + \bar{r}^{+}_{z,n} \right) \qquad \qquad \forall z,t,n
\end{align}

\paragraph{Electricity storages}
Charging and discharging of storages is constrained by the storages' power capacity $\bar{s}^{in}_{z,k}$ and $\bar{s}^{out}_{z,k}$, respectively. Similarly, the total energy that can be stored is constrained by the storage technology's energy capacity $\bar{v}_{z,k}$.
\begin{align}
s^{out}_{z,t,k} \leq \bar{s}^{out}_{z,k} + \bar{s}^{+}_{z,k} \qquad \qquad \forall z,t,k \\
s^{in}_{z,t,k} \leq \bar{s}^{in}_{z,k} + \bar{s}^{+}_{z,k} \qquad \qquad \forall z,t,k \\
v_{z,t,k} \leq \bar{v}^{0}_{z,k} + \bar{v}^{+}_{z,k} \qquad \qquad \forall z,t,k \\
\end{align}
Storage operation is subject to a storage balance, such that the current energy content must be equal to the previous period's energy content plus all energy flowing into the storage less all energy flowing out of the storage.
\begin{align}
v_{z,t,k} = v_{z,t-1,k} + \rho_{z,t,k} + \eta^{in}_{z,k} \: s^{in}_{z,t,k} - (\eta^{out}_{z,k})^{-1} \: s^{out}_{z,t,k} \qquad \qquad \forall z,t,k
\end{align}
Since the model can add storage power capacity and energy capacity independently, we require a storage to hold at least as much energy as it could store in (or out) in one hour.
\begin{align}
\bar{v}^{+}_{z,k} \geq \bar{s}^{+}_{z,k} \qquad \qquad \forall z,k
\end{align}

\paragraph{Electricity exchange}
Implicitly, \emph{medea} assumes that there are no transmission constraints within market zones. 
However, electricity exchange between market zones is subject to several constraints.

First, exchange between market zones is constrained by available transfer capacities. Transfer capacities can be expanded at constant, specific investment cost (see equation (\ref{transmission_expansion_cost})). This rules out economies of scale in transmission investment that might arise in interconnected, meshed grids.
\begin{align}
x_{z,zz,t} \leq \bar{x}^{0}_{z,zz} + \bar{x}^{+}_{z,zz} \qquad \qquad \forall z, zz, t \\
x_{z,zz,t} \geq - \left( \bar{x}^{0}_{z,zz} + \bar{x}^{+}_{z,zz} \right) \qquad \qquad \forall z, zz, t
\end{align}
By definition, electricity net exports $x_{z,zz,t}$ from $z$ to $zz$ must equal electricity net imports of $zz$ from $z$.
\begin{align}
x_{z,zz,t} = -x_{zz,z,t} \qquad \qquad \forall z, zz, t
\end{align}
Added transmission capacities can be used in either direction.
\begin{align}
\bar{x}^{+}_{z,zz} = \bar{x}^{+}_{zz,z} \qquad \qquad \forall z, zz
\end{align}
Finally, electricity cannot flow between zones where there is no transmission infrastructure in place (including intra-zonal flows).
\begin{align}
x_{z,zz,t} = 0 \qquad \qquad \forall z \notin NTC, zz \notin NTC, t \\
x_{zz,z,t} = 0 \qquad \qquad \forall z \notin NTC, zz \notin NTC, t
\end{align}

\paragraph{Decommissioning of thermal units}
Keeping a plant available for dispatch gives rise to quasi-fixed operation and maintenance costs. 
Such cost can be avoided by decommissioning an energy generator. This is modelled as a reduction in generation capacity, which cannot exceed installed capacity.
\begin{align}
\bar{g}^{-}_{z,i} \leq \bar{g}^{0}_{z,i} + \bar{g}^{+}_{z,i} \qquad \qquad \forall z,i
\end{align}

\paragraph{Ancillary services}
Power systems require various system services for secure and reliable operation, such as balancing services or voltage support through the provision of reactive power. Such system services can only be supplied by operational generators. Thus, we approximate system service provision by a requirement on the minimal amount of spinning reserves operating at each hour. 
We assume that ancillary services are supplied by conventional (thermal) power plants, hydro power plants, and storages.
\begin{align}
g_{z,t,i,\text{el}} + r_{z,t,\text{ror}} + s^{out}_{z,t,k} + s^{in}_{z,t,k} \geq \sigma_{z} \qquad \qquad \forall z,t,i,k
\end{align}

\paragraph{Curtailment}
Electricity generated from intermittent sources can be curtailed (disposed of) without any further cost (apart from implicit opportunity cost). 
\begin{align}
\Omega^{+}_{z,t} \leq \sum_{n} r_{z,t,n} \qquad \qquad \forall z, t
\end{align}

\end{document}