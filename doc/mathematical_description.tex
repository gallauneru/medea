\documentclass[11pt,a4paper]{article}
\usepackage[utf8]{inputenc}
\usepackage{amsmath}
\usepackage{amsfonts}
\usepackage{amssymb}
\DeclareMathOperator*{\argmax}{arg\,max}

\usepackage{natbib}
\usepackage{makecell}
\usepackage[gen]{eurosym} 
\usepackage[version=4]{mhchem}
%\usepackage{courier} 
\usepackage[inner=2.5cm,outer=2.5cm, bottom=3cm]{geometry}
\usepackage{ctable}
\usepackage{longtable}


\title{Description of the power system model \emph{medea}}
\date{\today}
\author{Sebastian Wehrle\thanks{Over recent years, \emph{medea} has grown into its current state thanks to inspiration and contributions from many friends and colleagues. 
In particular, I'd like to thank Johannes Schmidt, who contributed substantial parts of the initial model code, for getting \emph{medea} started and for keeping us running; Dieter Mayr and Stefan Höltinger for early contributions; Peter Regener for inspiration and endurance with all things programming. Current work on \emph{medea} is funded through grateful support from the European Research Council ("reFUEL" ERC-2017-STG 758149)} \\University of Natural Resources and Life Sciences, Vienna}

\begin{document}
\maketitle
\section{Overview}
\emph{medea} is a simple, stylized and parsimonious model of interconnected power and heating systems in Western and Central Europe.
It simulates investment in intermittent and conventional electricity and heat generation technologies as well as in cross-border electricity transmission capacities.
At the same time, the model determines the system-cost minimizing hourly dispatch of electricity and heat generators to meet price-inelastic demand.
Model results include hourly energy generation by technology and the associated fuel use and CO2 emissions, investment in and decommissioning of conventional and renewable generators and energy storages, hourly cross-border flows of electricity and potentially required transmission capacity expansion, as well as producer and consumer surplus.

A detailed description of the model is provided in the following. Section \ref{sets} gives an overview of the sets and set elements used in \emph{medea}. Sections \ref{parameters} and \ref{variables} introduce the model's parameters and variables, while section \ref{mathmodel} gives a detailed description of the model's mathematical formulation.

\newpage

\section{Sets} \label{sets}
Sets are denoted by upper-case latin letters, while set elements are denoted by lower-case latin letters.

\begin{longtable}{p{3.6cm} c c p{6cm}}
%\caption{Sets} \\ 
\hline
\textbf{name} & \makecell[l]{\textbf{math} \\ \textbf{symbol}} & \makecell[l]{\textbf{GAMS} \\\textbf{symbol}} & \textbf{elements} \\
\hline \hline
fuels & $f \in F$ & \texttt{f} & \texttt{nuclear, lignite, coal, gas, oil, biomass, power} \\ \hline
power generation technologies & $i \in I$ & \texttt{i} & 
\texttt{nuc, lig\_stm, lig\_stm\_chp, lig\_boa, lig\_boa\_chp,coal\_sub, coal\_sub\_chp, coal\_sc, coal\_sc\_chp, coal\_usc, coal\_usc\_chp, coal\_igcc, ng\_stm, 
ng\_stm\_chp, ng\_ctb\_lo, ng\_ctb\_lo\_chp, ng\_ctb\_hi, ng\_ctb\_hi\_chp, ng\_cc\_lo, ng\_cc\_lo\_chp, ng\_cc\_hi, ng\_cc\_hi\_chp, ng\_mtr, ng\_mtr\_chp, 
ng\_boiler\_chp, oil\_stm, oil\_stm\_chp, oil\_ctb, oil\_ctb\_chp, oil\_cc, oil\_cc\_chp, bio, bio\_chp, heatpump\_pth} \\ \hline
power to heat technologies & $h \in H \subset I$ & \texttt{h(i)} & \texttt{heatpump\_pth}  \\ \hline
CHP technologies & $j \in J \subset I$ & \texttt{j(i)} & \texttt{lig\_stm\_chp, lig\_boa\_chp, coal\_sub\_chp, coal\_sc\_chp, coal\_usc\_chp, ng\_stm\_chp, 
ng\_ctb\_lo\_chp, ng\_ctb\_hi\_chp, ng\_cc\_lo\_chp, ng\_cc\_hi\_chp, ng\_mtr\_chp, ng\_boiler\_chp, oil\_stm\_chp, oil\_ctb\_chp, oil\_cc\_chp, bio\_chp} \\ \hline
storage technologies & $k \in K$ & \texttt{k} & \texttt{res\_day, res\_week, res\_season, psp\_day, psp\_week, psp\_season, battery} \\ \hline
feasible operation region limits & $l \in L$ & \texttt{l} & \texttt{l1, l2, l3, l4} \\ \hline
energy products & $m \in M$ & \texttt{m} & \texttt{el, ht} \\ \hline
intermittent generators & $n \in N$ & \texttt{n} & \texttt{wind\_on, wind\_off, pv, ror}\\ \hline
time periods (hours) & $t \in T$ & \texttt{t} & \texttt{t1, t2, \ldots, t8760} \\ \hline
market zones & $z \in Z$ & \texttt{z} & \texttt{AT, DE} \\ \hline \hline
\end{longtable}

\section{Parameters} \label{parameters}
Parameters are denoted either by lower-case greek letters or by upper-case latin letters.

\begin{longtable}{p{4.1cm} c l c}
\hline
	\textbf{name} & \makecell[l]{\textbf{math} \\ \textbf{symbol}} & \makecell[l]{\textbf{GAMS} \\\textbf{symbol}} & \textbf{unit} \\ \hline \hline
% LOWER-CASE GREEK LETTERS
	distance between countries' center of gravity & $\delta_{z,zz}$ & \texttt{DISTANCE(z,zz)} & km \\ \hline
	fuel emission intensity & $\varepsilon_{f}$ & \texttt{CO2\_INTENSITY(f)} & $\text{t}_{\ce{CO2}}$ / MWh \\ \hline
	power plant efficiency & $\eta_{i,m,f}$ & \texttt{EFFICIENCY\_G(i,m,f)} & MWh / MWh \\ \hline
	discharging efficiency & $\eta^{out}_{z,k}$ & \makecell[l]{\texttt{EFFICIENCY\_S\_OUT(k)}} &  \\ \hline
	charging efficiency & $\eta^{in}_{z,k}$ & \makecell[l]{\texttt{EFFICIENCY\_S\_IN(k)}} &  \\ \hline	
	scaling factor for peak load & $\lambda_{z}$ & \texttt{LAMBDA(z)} &  \\ \hline
	value of lost load & $\mu_{z}$ & \texttt{VALUE\_NSE(z)} & \EUR / MWh \\ \hline
	inflows to storage reservoirs & $\rho_{z,t,k}$ & \makecell[l]{\texttt{INFLOWS(z,t,k)}} & MW \\ \hline	
	scaling factor for peak intermittent generation & $\sigma_{z}$ & \texttt{SIGMA(z)} &  \\ \hline
	intermittent generation profile & $\phi_{z,t,n}$ & \texttt{GEN\_PROFILE(z,t,n)} & $[0,1]$ \\ \hline
	peak intermittent generation profile & $\widehat{\phi}_{z,n}$ & \texttt{PEAK\_PROFILE(z,n)} & $[0,1]$ \\ \hline	
	inputs of feasible operating region & $\chi_{i,l,f}$ & \texttt{FEASIBLE\_INPUT(i,l,f)} & $[0,1]$ \\ \hline
	output tuples of feasible operating region & $\psi_{i,l,m}$ & \texttt{FEASIBLE\_OUTPUT(i,l,m)} & $[0,1]$ \\ \hline
% UPPER-CASE LATIN LETTERS	
	capital cost of intermittent generators (specific, annuity) & $C^{r}_{z,n}$ & \texttt{CAPITALCOST\_R(z,n)} & k\EUR / GW \\ \hline
	capital cost of thermal generators (specific, annuity) & $C^{g}_{z,i}$ & \texttt{CAPITALCOST\_G(z,i)} & k\EUR / GW \\ \hline
	capital cost of storages - power (specific, annuity) & $C^{s}_{z,k}$ & \makecell[l]{\texttt{CAPITALCOST\_S(z,k)}} & k\EUR / GW \\ \hline
	capital cost of storages - energy (specific, annuity) & $C^{v}_{z,k}$ & \makecell[l]{\texttt{CAPITALCOST\_V(z,k)}} & k\EUR / GW \\ \hline
	capital cost of transmission capacity (specific, annuity) & $C^{x}$ & \texttt{CAPITALCOST\_X} & k\EUR / GW \\ \hline
	energy demand & $D_{z,t,m}$ & \texttt{DEMAND(z,t,m)} & GW \\ \hline
	peak demand & $\widehat{D}_{z,m}$ & \texttt{PEAK\_LOAD(z,m)} & GW \\ \hline
	initial capacity of dispatchable generators & $\widetilde{G}_{z,i}$ & \texttt{INITIAL\_CAP\_G(z,tec)} & GW \\ \hline
\pagebreak
%\caption[Parameters]{(continued)}\\ 
\hline
	% UPPER-CASE LATIN LETTERS (CONTINUED)
	\textbf{name} & \makecell[l]{\textbf{math} \\ \textbf{symbol}} & \makecell[l]{\textbf{GAMS} \\\textbf{symbol}} & \textbf{unit} \\ \hline \hline
	variable O\&M cost of dispatchable generators & $O^{g}_{i}$ & \texttt{OM\_COST\_G\_VAR(i)} & \EUR / MWh \\ \hline
	variable O\&M cost of intermittent generators & $O^{r}_{z,n}$ & \texttt{OM\_COST\_R\_VAR(z,n)} & \EUR / MWh \\ \hline	
	\ce{CO2} price & $P^{e}_{t,z}$ & \texttt{PRICE\_CO2(t,z)} & \EUR / $\text{t}_{\ce{CO2}}$ \\ \hline
	fuel price & $P_{t,z,f}$ & \texttt{PRICE\_FUEL(t,z,f)} & \EUR / MWh \\ \hline
	quasi-fixed O\&M cost of dispatchable generators & $Q^{g}_{i}$ & \texttt{OM\_COST\_G\_QFIX(i)} & k\EUR / GW \\ \hline
	quasi-fixed O\&M cost of intermittent generators & $Q^{r}_{z,n}$ & \texttt{OM\_COST\_R\_QFIX(z,n)} & k\EUR / GW \\ \hline	
	initial capacity of intermittent generators & $\widetilde{R}_{z,n}$ & \texttt{INITIAL\_CAP\_R(z,n)} & GW \\ \hline
	initial discharging capacity of storages & $\widetilde{S}^{out}_{z,k}$ & \makecell[l]{\texttt{INITIAL\_CAP\_S\_OUT(z,k)}} & GW \\ \hline
	initial charging capacity of storages & $\widetilde{S}^{in}_{z,k}$ & \makecell[l]{\texttt{INITIAL\_CAP\_S\_IN(z,k)}} & GW \\ \hline
	initial energy storage capacity & $\widetilde{V}_{z,k}$ & \makecell[l]{\texttt{INITIAL\_CAP\_V(z,k)}} &  \\ \hline
	initial transmission capacity & $\widetilde{X}_{z,zz}$ & \texttt{INITIAL\_CAP\_X(z,zz)} & GW \\ \hline \hline	
\end{longtable}

\newpage


\section{Variables} \label{variables}
Variables are denoted by lower-case latin letters.

\begin{longtable}{p{5.6cm} c l c}
%\caption{Variables}\label{variables} \\ 
\hline
\textbf{name} & \makecell[l]{\textbf{math} \\ \textbf{symbol}} & \makecell[l]{\textbf{GAMS} \\\textbf{symbol}} & \textbf{Unit} \\ \hline \hline
fuel burn for energy generation & $b_{z,t,i,f}$ & \texttt{b(z,t,i,f)} & GW \\ \hline
total system cost & $c$ & \texttt{cost\_system} & k\EUR \\ \hline
zonal system cost & $c_{z}$ & \texttt{cost\_zonal(z)} & k\EUR \\ \hline
fuel cost & $c^{b}_{z,t,i}$ & \texttt{cost\_fuel(z,t,i)} & k\EUR \\ \hline
emission cost & $c^{e}_{z,t,i}$ & \texttt{cost\_co2(z,t,i)} & k\EUR \\ \hline
total o\&m cost of dispatchable generators & $c^{om}_{z,i}$ & \texttt{cost\_om\_g(z,i)} & k\EUR \\ \hline
total o\&m cost of intermittent generators & $c^{om}_{z,n}$ & \texttt{cost\_om\_r(z,n)} & k\EUR \\ \hline
capital cost of generators & $c^{g}_{z}$ & \texttt{cost\_invest\_g(z)} & k\EUR \\ \hline
total cost of non-served load & $c^{q}_{z}$ & \texttt{cost\_nse(z)} & k\EUR \\ \hline
capital cost of intermittent generators & $c^{r}_{z}$ & \texttt{cost\_invest\_r(z)} & k\EUR \\ \hline
capital cost of storages & $c^{s,v}_{z}$ & \texttt{cost\_invest\_sv(z)} & k\EUR \\ \hline
capital cost of interconnectors & $c^{x}_{z}$ & \texttt{cost\_invest\_x(z)} & k\EUR \\ \hline
\ce{CO2} emissions & $e_{z}$ & \texttt{emission\_co2(z)} & t \ce{CO2} \\ \hline
added capacity of dispatchables & $\widetilde{g}^{+}_{z,i}$ & \texttt{add\_g(z,i)} & GW \\ \hline
decommissioned capacity of dispatchables & $\widetilde{g}^{-}_{z,i}$ & \texttt{deco\_g(z,i)} & GW \\ \hline
energy generated by conventionals & $g_{z,t,i,m}$ & \texttt{g(z,t,i,m)} & GW \\ \hline
curtailed energy & $q^{+}_{z,t}$ & \texttt{q\_curtail(z,t)} & GW \\ \hline
non-served energy & $q^{-}_{z,t,m}$ & \texttt{q\_nse(z,t,m)} & GW \\ \hline
added capacity of intermittents & $\widetilde{r}^{+}_{z,n}$ & \texttt{add\_r(z,n)} & GW \\ \hline
decommissioned capacity of intermittents & $\widetilde{r}^{-}_{z,n}$ & \texttt{deco\_r(z,n)} & GW \\ \hline
electricity generated by intermittents & $r_{z,t,n}$ & \texttt{r(z,t,n)} & GW \\ \hline
added storage capacity (power) & $\widetilde{s}^{+}_{z,k}$ & \texttt{add\_s(z,k)} & GW \\ \hline
energy stored in & $s^{in}_{z,t,k}$ & \texttt{s\_in(z,t,k)} & GW \\ \hline
energy stored out & $s^{out}_{z,t,k}$ & \texttt{s\_out(z,t,k)} & GW \\ \hline
added storage capacity (energy) & $\widetilde{v}^{+}_{z,k}$ & \texttt{add\_v(z,k)} & GWh \\ \hline
storage energy content & $v_{z,t,k}$ & \texttt{v(z,t,k)} & GWh \\ \hline
operating region weight & $w_{z,t,i,l}$ & \texttt{w(z,t,i,l)} & \\ \hline
added transmission capacity & $\widetilde{x}^{+}_{z,zz}$ & \texttt{add\_x(z,zz)} & GW \\ \hline
electricity net export & $x_{z,zz,t}$ & \texttt{x(z,zz,t)} & GW \\ \hline \hline
\end{longtable}

\newpage


\section{Naming system}
\ctable[
	maxwidth = 160mm,
	pos = h]
	{X p{2.0cm} p{2.0cm} p{2.0cm} p{2.0cm} p{2.0cm}}
	{
	\tnote[$\dagger$]{parameter} 
	\tnote[$\ddagger$]{variable}
	}
	{\FL 
	 &  initial capacity\tmark[$\dagger$] & added capacity\tmark[$\ddagger$] & \makecell[l]{decom-\\missioned \\capacity\tmark[$\ddagger$]} & \makecell[l]{specific \\investment \\cost\tmark[$\dagger$]} & dispatch\tmark[$\ddagger$] \ML
	thermal units & $\widetilde{G}_{z,i}$ & $\widetilde{g}^{+}_{z,i}$ & $\widetilde{g}^{-}_{z,i}$ & $C^{g}_{z,i}$ & $g_{z,t,i,m}$ \ML
	\makecell[l]{intermittent \\units} &  $\widetilde{R}_{z,n}$ & $\widetilde{r}^{+}_{z,n}$ & $\widetilde{r}^{-}_{z,n}$ & $C^{r}_{z,n}$ & $r_{z,t,n}$ \ML
	\makecell[l]{storages \\(power)} & $\widetilde{S}_{z,k}$ & $\widetilde{s}^{+}_{z,k}$ & na & $C^{s}_{z,k}$ & $s_{z,t,k}$ \ML  % $\widetilde{s}^{-}_{z,k}$
	\makecell[l]{storages \\(energy)} & $\widetilde{V}_{z,k}$ & $\widetilde{v}^{+}_{z,k}$ & na & $C^{v}_{z,k}$ & na \ML  % $\widetilde{v}^{-}_{z,t,k}$
	transmission &  $\widetilde{X}_{z,zz}$ & $\widetilde{x}^{+}_{z,zz}$ & na & $C^{x}_{z,zz}$ & $x_{z,zz,t}$ \LL  % $\widetilde{x}^{-}_{z,zz}$
}

\newpage
\section{Mathematical description} \label{mathmodel}

\paragraph{Model objective}
\emph{medea} minimizes total system cost $c$, i.e. the total cost of generating electricity and heat from technologies and capacities adequate to meet demand, over a large number  of decision variables, essentially representing investment and dispatch decisions in each market zone $z$ of the modelled energy systems.
\begin{align}
\min c = \sum_{z} (c_{z})
\end{align}
Zonal system costs $c_{z}$ are the sum of fuel cost $c^{b}_{z,t,i}$, emission cost $c^{e}_{z,t,i}$, operation and maintenance cost, capital costs of investment in conventional and intermittent generation ($c^{g}_{z}$, $c^{r}_{z}$), storage ($c^{s,v}_{z}$) and transmission ($c^{x}_{z}$) equipment, and the cost of non-served load ($c^{q}_{z}$) that accrues when demand is not met, e.g. when there is a power outage. 
\begin{align}
%\begin{split}
c_{z} = \sum_{t,i}  c^{b}_{z,t,i} + \sum_{t,i} c^{e}_{z,t,i} + \sum_{i} c^{om}_{z,i} + \sum_{n} c^{om}_{z,n} + c^{g}_{z} + 
 c^{r}_{z} + c^{s,v}_{z} + c^{x}_{z} + c^{q}_{z} \qquad \qquad \forall z
%\end{split}
\end{align}
The components of zonal system costs are calculated as given in equations (\ref{fuel_cost}) to (\ref{lost_load_cost}).
Lower-case $c$ represent total cost, while upper-case $C$ denotes specific, annualized capital cost of technology investment. Prices for fuels and \ce{CO2} are denoted by $P$.
\begin{align}
&c^{b}_{z,t,i}& =&\ \sum_{f} \left( P_{t,z,f} \: b_{t,z,i,f} \right) \qquad \qquad &\forall z,t,i \label{fuel_cost} \\
&c^{e}_{z,t,i}& =&\ \sum_{f} \left( P^{e}_{t,z} \: e_{z,t,i} \right) \qquad \qquad &\forall z,t,i\\
&c^{om}_{z,i}& =&\ Q^{g}_{i} \left(\widetilde{G}_{z,i} - \widetilde{g}^{-}_{z,i} + \widetilde{g}^{+}_{z,i}\right) + \sum_{t,m} \left(O^{g}_{i} \: g_{z,t,i,m}\right) \qquad \qquad &\forall z,i \\
&c^{om}_{z,n}& =&\ Q^{r}_{n} \left(\widetilde{R}_{z,n} - \widetilde{r}^{-}_{z,n} + \widetilde{r}^{+}_{z,n}\right) + \sum_{t} \left(O^{r}_{n} \: r_{z,t,n}\right) \qquad \qquad &\forall z,n \\
&c^{g}_{z}& =&\ \sum_{i} \left( C^{g}_{z,i} \: \widetilde{g}^{+}_{z,i} \right) \qquad \qquad &\forall z\\
&c^{r}_{z}& =&\ \sum_{n} \left( C^{r}_{z,n} \: \widetilde{r}^{+}_{z,n} \right) \qquad \qquad &\forall z\\
&c^{s,v}_{z}& =&\ \sum_{k} \left( C^{s}_{z,k} \: \widetilde{s}^{+}_{z,k} + C^{v}_{z,k} \: v^{+}_{z,k} \right) \qquad \qquad &\forall z\\
&c^{x}_{z}& =&\ \frac{1}{2} \: \sum_{zz} (C^{x} \: \delta_{z,zz} \: \widetilde{x}^{+}_{z,zz}) \qquad \qquad &\forall z \label{transmission_expansion_cost}\\
&c^{q}_{z}& =&\ \mu \sum_{t,m} q^{-}_{z,t,m} \qquad \qquad &\forall z \label{lost_load_cost}
\end{align}

\paragraph{Market clearing}
In each hour, the markets for electricity and heat have to clear.
Equation (\ref{market_clearing_el}) ensures that the total supply from conventional and intermittent sources, and storages equals total electricity demand plus net exports, electricity stored and used for heat generation. 
Likewise, equation (\ref{market_clearing_ht}) clears the heat market by equating heat generation to heat demand.
\begin{align}
\begin{split}
\sum_{i} g_{z,t,i,\text{el}} + \sum_{k} s^{out}_{z,t,k} + \sum_{n} r_{z,t,n} &=  \\ D_{z,t,\text{el}} + \sum_{i} b_{z,t,i,\text{el}} + & \sum_{k} s^{in}_{z,t,k} + \sum_{zz} x_{z,zz,t} - q^{-}_{z,t,\text{el}} + q^{+}_{z,t} \qquad \forall z,t 
\end{split}
\label{market_clearing_el}
\end{align} 
\begin{align}
\sum_{i} g_{z,t,i,\text{ht}} = D_{z,t,\text{ht}} - q^{-}_{z,t,\text{ht}} \qquad \forall z,t \label{market_clearing_ht}
\end{align}
\emph{medea} can be thought of as representing energy-only electricity and heat markets without capacity payments. Then, the marginals of the market clearing equations (\ref{market_clearing_el}) and (\ref{market_clearing_ht}), $\partial C / \partial D_{z,t,m}$, can be interpreted as the zonal prices for electricity and heat, respectively.

\paragraph{Energy generation}
Energy generation $g_{z,t,i,m} \geq 0$ is constrained by available installed capacity, which can be adjusted through investment ($\widetilde{g}^{+}_{z,i} \geq 0$) and decommissioning $\widetilde{g}^{-}_{z,i} \geq 0$.
\begin{align}
g_{z,t,i,m} \leq \widetilde{G}_{z,i} + \widetilde{g}^{+}_{z,i} - \widetilde{g}^{-}_{z,i} \qquad \qquad \forall z,t,i,m
\end{align}
Generator efficiency $\eta$ determines the amount of fuel $b_{z,t,i,f} \geq 0$ that needs to be spent in order to generate a given amount of energy.
\begin{align}
g_{z,t,i,m} \leq \sum_{f} \eta_{i,m,f} \: b_{z,t,i,f} \qquad \qquad \forall z,t,i \notin J
\end{align}

\paragraph{Thermal co-generation}
Co-generation units jointly generate heat and electricity. All feasible combinations of heat and electricity generation along with the corresponding fuel requirement are reflected in so-called `feasible operating regions'. 
The elements $l \in L$ span up a three-dimensional, convex feasible operating region for each co-generation technology. 
The weights $w_{z,t,i,l} \geq 0$ form a convex combination of the corners $l$, which are scaled to the available installed capacity of each co-generation technology.
Heat and electricity output along with the corresponding fuel requirement is then set according to the chosen weights.
\begin{align}
\sum_{l} w_{z,t,i,l} = \widetilde{G}_{z,i} + \widetilde{g}^{+}_{z,i} - \widetilde{g}^{-}_{z,i} \qquad \qquad \forall z,t,i \in J \\
g_{z,t,i,m} \leq \sum_{l} \psi_{i,l,m} \: w_{z,t,i,l} \qquad \qquad \forall z,t,i \in J, m \\
b_{z,t,i,f} \geq \sum_{l} \chi_{i,l,f} \: w_{z,t,i,l} \qquad \qquad \forall z,t,i \in J, f
\end{align}

\paragraph{Intermittent electricity generation}
Electricity generation from intermittent sources wind (on-shore and off-shore), solar irradiation, and river runoff follows generation profiles $\phi_{z,t,n} \in [0,1]$ and is scaled according to corresponding installed ($ \widetilde{R}_{z,n}$) and added ($\widetilde{r}^{+}_{z,n} \geq 0$) capacity.
\begin{align}
r_{z,t,n} = \phi_{z,t,n} \: \left( \widetilde{R}_{z,n} - \widetilde{r}^{-}_{z,n} + \widetilde{r}^{+}_{z,n} \right) \qquad \qquad \forall z,t,n
\end{align}

\paragraph{Electricity storages}
Charging ($s^{in}_{z,t,k} \geq 0$) and discharging ($s^{out}_{z,t,k} \geq 0$) of storages is constrained by the storages' installed ($\widetilde{S}^{in}_{z,k}, \widetilde{S}^{out}_{z,k}$) and added ($\widetilde{s}^{+}_{z,k} \geq 0$) charging and discharging power, respectively. Similarly, the total energy that can be stored is constrained by the storage technology's initial ($\widetilde{V}_{z,k}$) and added ($\widetilde{v}^{+}_{z,k} \geq 0$) energy capacity.
\begin{align}
s^{out}_{z,t,k} \leq \widetilde{S}^{out}_{z,k} + \widetilde{s}^{+}_{z,k} \qquad \qquad \forall z,t,k \\
s^{in}_{z,t,k} \leq \widetilde{S}^{in}_{z,k} + \widetilde{s}^{+}_{z,k} \qquad \qquad \forall z,t,k \\
v_{z,t,k} \leq \widetilde{V}_{z,k} + \widetilde{v}^{+}_{z,k} \qquad \qquad \forall z,t,k
\end{align}
Storage operation is subject to a storage balance, such that the current energy content must be equal to the previous period's energy content plus all energy flowing into the storage less all energy flowing out of the storage.
\begin{align}
v_{z,t,k} = \rho_{z,t,k} + \eta^{in}_{z,k} \: s^{in}_{z,t,k} - (\eta^{out}_{z,k})^{-1} \: s^{out}_{z,t,k} + v_{z,t-1,k} \qquad \qquad \forall z,t,k: t>1, \: \eta^{out}_{z,k} > 0
\end{align}
Since the model can add storage power capacity and energy capacity independently, we require a storage to hold at least as much energy as it could store in (or out) in one hour.
\begin{align}
\widetilde{v}^{+}_{z,k} \geq \widetilde{s}^{+}_{z,k} \qquad \qquad \forall z,k
\end{align}

\paragraph{Emission accounting}
Burning fossil fuels for energy generation produces emissions of carbon dioxide (\ce{CO2}). The amount of \ce{CO2} emitted is tracked by the following equation
\begin{align}
e_{z,t,i} = \sum_{f} \left( \varepsilon_{f} \: b_{z,t,i,f} \right) \qquad \qquad \forall z,t,i
\end{align}

\paragraph{Electricity exchange}
Implicitly, \emph{medea} assumes that there are no transmission constraints within market zones. 
However, electricity exchange between market zones is subject to several constraints.

First, exchange between market zones is constrained by available transfer capacities. Transfer capacities can be expanded at constant, specific investment cost (see equation (\ref{transmission_expansion_cost})). This rules out economies of scale in transmission investment that might arise in interconnected, meshed grids.
\begin{align}
x_{z,zz,t} \leq \widetilde{X}_{z,zz} + \widetilde{x}^{+}_{z,zz} \qquad \qquad \forall z, zz, t \\
x_{z,zz,t} \geq - \left( \widetilde{X}_{z,zz} + \widetilde{x}^{+}_{z,zz} \right) \qquad \qquad \forall z, zz, t
\end{align}
By definition, electricity net exports $x_{z,zz,t}$ from $z$ to $zz$ must equal electricity net imports of $zz$ from $z$.
\begin{align}
x_{z,zz,t} = -x_{zz,z,t} \qquad \qquad \forall z, zz, t
\end{align}
Added transmission capacities can be used in either direction.
\begin{align}
\widetilde{x}^{+}_{z,zz} = \widetilde{x}^{+}_{zz,z} \qquad \qquad \forall z, zz
\end{align}
Finally, electricity cannot flow between zones where there is no transmission infrastructure in place (including intra-zonal flows).
\begin{align}
x_{z,zz,t} = 0 \qquad \qquad \forall z, zz, t: \widetilde{X}_{z,zz} = 0 \\
x_{zz,z,t} = 0 \qquad \qquad \forall z, zz, t: \widetilde{X}_{z,zz} = 0
\end{align}

\paragraph{Decommissioning}
Keeping plant available for generation gives rise to quasi-fixed operation and maintenance costs. 
Such cost can be avoided by decommissioning an energy generator. This is modelled as a reduction in generation capacity, which cannot exceed installed capacity.
\begin{align}
\widetilde{g}^{-}_{z,i} &\leq \widetilde{G}_{z,i} + \widetilde{g}^{+}_{z,i} \qquad \qquad \forall z,i \\
\widetilde{r}^{-}_{z,n} &\leq \widetilde{R}_{z,n} + \widetilde{r}^{+}_{z,n} \qquad \qquad \forall z,n
\end{align}

\paragraph{Ancillary services}
Power systems require various system services for secure and reliable operation, such as balancing services or voltage support through the provision of reactive power. Such system services can only be supplied by operational generators. Thus, we approximate system service provision by a requirement on the minimal amount of spinning reserves operating at each hour. 
We assume that ancillary services are supplied by conventional (thermal) power plants, hydro power plants, and storages.
The requirement for spinning reserves is proportional to electricity peak load $\widehat{D}_{z,\text{el}} = \max_{t} D_{z,t,\text{el}}$ and peak generation from wind and solar resources, where $\widehat{\phi}_{z,n} = \max_{t} \phi_{z,t,n}$.
\begin{align}
\sum_{i}\left( g_{z,t,i,\text{el}} \right) + r_{z,t,\text{ror}} + \sum_{k} \left( s^{out}_{z,t,k} + s^{in}_{z,t,k} \right) \geq \lambda_{z} \widehat{D}_{z,\text{el}} + \sigma_{z} \sum_{n \setminus \{ \text{ror}\}} \widehat{\phi}_{z,n} (\widetilde{R}_{z,n} + \widetilde{r}^{+}_{z,n}) \qquad \forall z,t
\end{align}

\paragraph{Curtailment}
Electricity generated from intermittent sources can be curtailed (disposed of) without any further cost (apart from implicit opportunity cost). 
\begin{align}
q^{+}_{z,t} \leq \sum_{n} r_{z,t,n} \qquad \qquad \forall z, t
\end{align}

\end{document}